/\chapter{Introduction to Selected Gesture Systems} % (fold)
\label{cha:Introduction to Selected Gesture Systems}

\section{Selecting Gestures to Test}

To keep the implementation and study time in a reasonable timeframe, only a limited number of locomotion systems can be picked for the comparative study. Also to keep the systems comparable, only teleportation based systems are going to be considered since teleportation is currently the most popular and accessible method. This leaves the gestures from the games industry and the gestures researched by Schäfer et al. \cite{Schafer2021}. 

The games industry has produced gesture systems for a variety of environments and use cases. The vacation simulator pull gesture is only used for low resolution tasks and is therefore too limited. The telepath system is very interesting but harder to compare to the other systems. This would have made it difficult to implement and evaluate. %cite
The elixir gesture system on the other hand will be implemented for the study. A version of it was also proposed by the participants of the Gesture Evaluation Study. This system will be referred to as triangle gesture in the following work. %cite

The gestures from the Schäfer study have variations for both two and one handed modes. The second hand is only used to confirm the teleport and not for the targeting system. 
This is interesting when compared to the study performed by Caggianese et al. \cite{Caggianese} where the same one handed gestures are used only with a confirmation method using the same hand. Schäfer et al. use a time delay of $1.5$ seconds to confirm the one handed versions of their gestures. According to the participants of the Gesture Evaluation Study, this might lead to unexpected teleport jumps and a confirmation step would be preferred. The confirmation mechanic developed by Caggianese et al. is proven to work well and could make the one handed index and palm methods of Schäfer more controllable. The gestures were therefore modified in this way for the implementation to reduce user discomfort and comply to the users intuitions better.

\section{Teleport Mechanic and Design}
During the presentation of the previous results in an expert interview, the question came up how improve the user experience of the targeting system that would convert a vector from the gesture into a teleport destination. In the games industry it is standard practice to use a ray-cast of some kind and a target reticle that shows the destination point. The ray-cast can be implemented linearly or by using a simulated projectile path. Both have some benefits and drawbacks, especially in this use-case since it might force the user to put their hands in an uncomfortable position to reach a target. It was proposed that the system should adapt to the user. This means that if the vector produced from the gesture recognition system hits a valid point in the world, this point will be used as a destination. However, if the vector does not hit the ground within a specified distance, the targeting system uses a simulated projectile path to curve the teleport path downwards. Both targeting versions have the same smoothing applied in order to eliminate small tracking errors or gesture changes.

