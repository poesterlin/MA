\chapter{Conclusion}

New technological leaps can be very exiting. The ability to capture the exact positions of every joint of a users hand opens up lots of possibilities as well as challenges. The goal of this evaluation was to be able to give a recommendation how the ideal gesture would look like that allows for comfortable, intuitive and accurate gesture based teleportation. However, it seems to be the case that there is more work to be done. 

After the excising literature was reviewed, a gesture elicitation study was conducted. This was done to find intuitive gestures that are proposed directly by the users themselves. This resulted in the three gestures shown again below. They all include a second confirmation step that was explicitly requested by the participants of the elicitation study. In the previous work this was not consistently implemented in the same way. There are also records of methods that use a fixed dwell time before confirming a teleportation destination. Also one hand used for targeting and the other for confirmation is a variation found in the related work. The participants of the study, however, gave a clear recommendation for a confirmation step. 

After the study was completed the resulting gestures where grouped into similar categories and some slight differences in the execution and the behavior of the gestures were adjusted so the resulting three gestures could be implemented as a prototype. It is using stored recordings of a base version of the gesture as a reference for the detection algorithm, together with a calibration procedure that corrects for the users hand size and stores individual variants of the gesture.

\begin{figure}[!h]
    \minipage{0.32\textwidth}
        \includegraphics[width=\linewidth]{figures/index tracker.jpg}
        \caption{Index gesture}
    \endminipage\hfill
    \minipage{0.32\textwidth}
        \includegraphics[width=\linewidth]{figures/palm tracker.jpg}
        \caption{Palm gesture}
    \endminipage
    \minipage{0.32\textwidth}
        \includegraphics[width=\linewidth]{figures/triangle tracker.jpg}
        \caption{Triangle gesture}
    \endminipage
    \hfill
    \caption{Gestures based on the gesture elicitation study and how they convert the gesture into a vector.}
\end{figure}

To investigate what effect the gestures have, while keeping everything else the same, a controlled lab study was performed with 18 participants completing two tasks. 

While there was not always clear evidence for this, the triangular gesture that uses both thumbs and index fingers to construct a triangle produces the best result when looking at the task completion times and the dependability the users reported in a UEQ questionnaire, as well as the teleport trajectories. The ranking done by the participants was even more mixed, however, so that no statistically significant differences could be found. 
