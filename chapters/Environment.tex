
\chapter{Study Environments}
In preparation for the user study, I created a detailed forest environment and a simple path integration test environment in Unity. Both environments already have a basic tutorial in place to inform the user how the gestures work using animations, this is not finished however and should also be tested before conducting the study.


\section{Forrest Environment}
The forest environment was created for the main part of the study. Here the user can teleport around freely in a large forest area. It requires at least five teleport jumps across and creates an immersive experience for the user. The application is running natively on the Oculus Quest headset without any connection to a PC that could impact the VR experience in a negative way. The application was optimized to run very well so it is possible to enable the highest hand-tracking polling rate. This makes the experience very natural and improves immersion. As a preparation, this environment was also used to record the gestures. For this, the Oculus "options" gesture was used to trigger the recording with a time delay. The recording functionality will be disabled in the final version of the application. % map display

During the study, the user is instructed to collect four magic potions from all over the map. Each one is hidden in a secret location on the map. The user has to search each one using a different teleportation technique and return the item to the starting location. There a magic pot waits for them to add all the ingredients needed for a spell. This mechanic is supposed to keep the interactions fun and immersive. Some sound effects and animations are included as well. Picking up the items works using a script from the OVR framework that uses a system of grabbers and grabbables. The potions are using the grabbable script while the hands are defined as grabbers with the pinch distance between the user's thumb and index finger as a trigger. In the bimanual gesture system, the potion can not be carried to the starting position since there is no free hand for the user available to carry the potion. In this case, the potion is clipped to a position representing the users' belt where it can be taken off to be used in the magic pot once the user has teleported back to the starting location. This interaction is taken from the Oculus Quest VR game Vader Immortal, where tools and a Star Wars Lightsaber can be attached to a virtual belt. 


\section{Path Integration Test Environment} % and accuracy?
To test how well the user can keep their path integration while teleporting, a simple test environment was created. The environment is kept very simple and only includes a white ground plane and the skybox. The user starts at point A and is given the task to teleport to point B that they are free to choose, to turn around and then return to point A. The environment was created to measure the distance between the actual point A and the point A' the user landed on after two teleportation steps. This distance is recorded together with the distance of the first teleport. The test is run with all gestures in a randomized order. The recorded data will later give insight into the ability to choose an accurate point and how well the path integration works while teleporting.