\chapter{Discussion} % (fold)
\label{cha:Discussion}

The quantitative results favour the triangle gesture since it produces the lowest task completion time, the lowest teleport delay as well als the cleanest teleport trajectories. 

\subsection{Future Work}

As part of the final quantitative questionnaire after all tasks were conducted, the participants were asked to come up with ideas for other gesture systems, similar to the gesture elicitation study. It was to be expected that the answers would be biased by the study up to this point but this task was included because this study included more participants that now also had some experience with how the teleportation using hand-tracking feels like. As expected, the answers were largely based on the previous gestures but some interesting ideas could still be collected. Multiple participants came up with the idea of a one handed version of the triangle gesture system. Also one participant suggested to change the palm gesture systems target direction so the targeting hand would be in a more comfortable position and the direction vector would not change when the hand is closed to confirm the teleport location. The change they proposed would have the hand in an orientation where the thumb would face towards the user with the hand edge pointing in the teleport direction. Both ideas seem like a promising implementation of the participants feedback and could be interesting to investigate in future research.

At the time of writing, a new update to the hand-tracking software of Meta Quest 2, the VR headset that was used to perform both studies, became available. According to the release notes and early reviews from developers, %TODO: cite
this update improves the tracking overall a lot but specifically helps with hand-on-hand interaction. Gestures that rely on this type of interaction could be an interesting target of further research since they were not technically feasible to implement for this work prior to the update. The participants of the gesture elicitation study proposed some gestures that could benefit from this update but not a lot of them. The participants could have been biased  by the current version of the tracking however, since they would experience the tracking cutting out if they tried to make gestures with hand-on-hand interaction as seen in figure \ref{fig:hands20}. Therefore it would be required to perform another gesture elicitation study to confirm if gestures with hand-on-hand interaction are not intuitive for the users. Also the general improvements to the tracking could benefit the usability of the system overall, which would also be an interesting comparison to make in the future. 

\begin{figure}[!ht]
    \centering
    \includegraphics[width=\textwidth]{figures/hands20.png}
    \caption{Update to hand-tracking allowing hand-on-hand interactions, while the previous implementation looses tracking.}
    \label{fig:hands20}
\end{figure}