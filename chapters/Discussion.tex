\chapter{Discussion} % (fold)
\label{cha:Discussion}

The quantitative results favour the triangle gesture since it produces the lowest task completion time and the lowest teleport delay. The usability results on the other hand are rarely conclusive, with the dependability being the only UEQ dimension with statistically significant differences between the gestures. All of the gestures are favoured by the participants in some dimension which makes the results hard to interpret. The index gesture for example is getting UEQ results that are better than expected. Users reported being frustrated by the false positive teleport actions but still reported the gesture to be most attractive, efficient, stimulating and novel. A reason for this can be seen in the comments given by the participants. The participants complained about the tracking but still reported the gesture to be fun which is not an attribute given to any of the other gestures. A hypotheses is therefore that the tracking and activation issues are compensated by the gesture being the most fun to use for the user. With improved tracking this could be solved and result in both good usability and good task completion times. For now with the current tracking algorithm, the triangle gesture produces the best results and is reported to be the most dependent. The triangle gesture has the technical benefit of being easy to track because none all of the important joints are visible to the tracking system. It also should have increased accuracy and stability since the algorithm is using four joint positions to generate the targeting vector. This gives the triangle gesture an advantage in some ways but has some drawbacks too. A mechanism had to be added to allow the users to carry an object to be able to complete the first task. This would not be a technical challenge but might be inappropriate for some use cases in real applications. Ideally a gesture for something as important as locomotion would be something that does fade into the background and something does not needs to be worked around for. This makes one-handed gestures more attractive to implement in applications. While the triangle gesture does not produce a significant difference in the overall load experienced by the participants, it is also not reported to be fun. It is the recommended gesture with the current setup but this is likely going to change as the technology improves and is worth further investigation. Applications that require very accurate but infrequent teleportation steps could still benefit from the triangle gesture over the index gesture even in the future but this makes the use-case very limited. Since a standard way to teleport using gestures has not been established it is worth to pick a gesture that is compatible with a variety of environments and tasks. Especially since gestures do not offer the affordance that buttons on a controller do and a user would always need a tutorial to show them how to teleport if there is no standard way established. 


\subsection{Future Work}

As part of the final quantitative questionnaire after all tasks were conducted, the participants were asked to come up with ideas for other gesture systems, similar to the gesture elicitation study. It was to be expected that the answers would be biased by the study up to this point but this task was included because this study included more participants that now also had some experience with how the teleportation using hand-tracking feels like. As expected, the answers were largely based on the previous gestures but some interesting ideas could still be collected. Multiple participants came up with the idea of a one handed version of the triangle gesture system. Also one participant suggested to change the palm gesture systems target direction so the targeting hand would be in a more comfortable position and the direction vector would not change when the hand is closed to confirm the teleport location. The change they proposed would have the hand in an orientation where the thumb would face towards the user with the hand edge pointing in the teleport direction. Both ideas seem like a promising implementation of the participants feedback and could be interesting to investigate in future research.

At the time of writing, a new update to the hand-tracking software of Meta Quest 2, the VR headset that was used to perform both studies, became available. According to the release notes and early reviews from developers, %TODO: cite
this update improves the tracking overall a lot but specifically helps with hand-on-hand interaction. Gestures that rely on this type of interaction could be an interesting target of further research since they were not technically feasible to implement for this work prior to the update. The participants of the gesture elicitation study proposed some gestures that could benefit from this update but not a lot of them. The participants could have been biased  by the current version of the tracking however, since they would experience the tracking cutting out if they tried to make gestures with hand-on-hand interaction as seen in figure \ref{fig:hands20}. Therefore it would be required to perform another gesture elicitation study to confirm if gestures with hand-on-hand interaction are not intuitive for the users. Also the general improvements to the tracking could benefit the usability of the system overall, which would also be an interesting comparison to make in the future. 

\begin{figure}[!ht]
    \centering
    \includegraphics[width=0.8\textwidth]{figures/hands20.png}
    \caption{Update to hand-tracking allowing hand-on-hand interactions, while the previous implementation looses tracking.}
    \label{fig:hands20}
\end{figure}