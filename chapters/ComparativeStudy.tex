\chapter{Comparative Study}
The user study to compare teleportation techniques and to provide quantitative figures is detailed in the following chapter. Using the detection methods from the previous chapters in with a virtual environment, a controlled lab study was performed. 

\section{}

\section{Study Design}



% Research question: 
% wie unterscheiden sich die 3 gesten, lernability, statisfaction
% How does the user experience differ between the 3 gestures?

% \section{Requirements}
% This section explains the requirements set to evaluate the gesture.
% For the study, the requirements have to be measurable. Therefor criteria for every requirement have to be set:

% \begin{enumerate}
%     \item The gesture can be recognized by the tracking system with a success rate of at least 90\%
%     \item The system is efficient and effective to use
%     \item No more than 20\% of users experience some cybersickness, no participant has to abort the test prematurely
%     \item Even after extended use, the gesture does create strain in over 10\% of participants
%     \item System has a SUS score of above 70.
%     \item The user can get a sense of the distance they covered
%     \item The system encourages walking in real life if possible and is not used to travel distances below 50cm
% \end{enumerate}

\section{Study Procedure}
The study consisted of two main experiments. It was conducted with 18 participants. The participants first got an introduction to the gesture system, succeeded by the calibration step explained in the implementation section. %ref
The order of the gesture systems the participants had to use was fully counter balanced in order to minimize the learning effect. This means that during the study it was made sure that three participants would always receive each of the six combinations possible. 

%TODO: ablauf schaubild
\subsection{Experiment One}
The first experiment is based on a task that would be a realistic use case for a locomotion system. It was conducted using a playful magical VR environment shown in ... %ref figure. 
The participant was tasked to collect an ingredient for a magic potion that could be hidden anywhere on the map. The teleportation system has to be used to search for the ingredient and for it to be brought back to the starting point. The user received a short in-game tutorial each time with a picture of the ingredient that should be found with animated hands presenting the gesture and how to use it. This was inspired by the tutorial in the ... game %cite
Each teleport was recorded, together with the players position and head gaze direction. The task was specifically designed not to put too much pressure on the participant so they can get to know the gesture and ideally have fun looking around. However, like in a real application, the teleportation was still an essential part of the experience and so small details about the implementation, the detection and the ergonomics would still stand out to the participant. The ingredients were randomly hidden in one out of eight predefined spots so the user would have to search in a new spot every time.

\subsection{Experiment Two}
The second experiment was based on the work from Bozgeyikli et al. %cite
The original design for the experiment was followed as closely as possible. There were however some changes that needed to be made to adapt it. Like before, the experiment starts by first showing an animated tutorial again. This time the user is instructed to teleport between platforms placed on the ground quickly. This time the environment is kept very simple, so the focus is on the task and so the user would not get distracted by the world. Unlike before, during this experiment the task execution time and the accuracy were the main focus. In order to record the start time correctly the original environment by Bozgeyikli et al. was modified and a button was placed in the center of the platforms. The participants were instructed to teleport to the button and press it once they felt ready to start. After the button was activated, it was removed from the environment and the first platform was highlighted. Like in the original experiment the next platform on the ground would light up to tell the user where to teleport to. To give the user feedback, a quick confirmation sound played when a teleport hit the correct platform. The player was instructed to stay on the platform for three seconds, after which a different sound was played and the next platform was highlighted. If a user accidentally teleported off the platform too soon, an error sound was played and they had to return to the platform and wait for the full three seconds. 

\section{Experiment Design}

\section{Apparatus}
%TODO: tablet questionnaire

The experiments were conducted using a Meta Quest 2 with hand-tracking enabled. The headset was connected to the internet and submitted all recorded data to a custom server. To be able to monitor and control the experiment, the server also had a web interface with a number of tools available. The study operator could control which experiment is currently running. During testing it was found that it can be disorienting for the participant to be switched from on scene to another. That is why this control was done manually so the operator can make sure the participant knows what is coming and is not surprised by the scene switch.

The operator would also manually run the calibration of each gesture. The calibration procedure is explained in ... % TODO: \ref{}
To check if the gesture was recorded correctly, the server provides a 3D visualization of the gesture, as well as a way to delete improperly recorded information as shown in \ref{fig:vis}. This visualization also shows which of the previous recordings are also able to be used to aid the gesture detection step, since they are similar to the current recording. They are shown in green in the visualization, while red points represent ignored data points. To be able to find which data points belong to which recording each can be selected individually and deleted if there was a problem. To be able to find outliers quickly, the average distance to all the other gestures is also displayed in the selection menu. This was very handy to delete a recording that was assigned to the wrong gesture state and therefore very dissimilar to the other recordings. The recording would then show a large distance to the others and could easily be deleted and rerecorded. All this was very important so the calibration was consistent and reliable for every participant.

\begin{figure}[!ht]
    \centering
    \includegraphics[width=0.8\textwidth]{figures/visualizer.jpg}
    \caption{3D visualization of recorded gesture information as a point cloud, together with static joints.}
    \label{fig:vis}
\end{figure}
% TODO: update grafik

As shown in \ref{fig:map}, the operator was also able to see the position and gaze direction of the participant on a map, as well as the position of the magic potion. This was done in order to be able to give some support to the participant if they required it and to check if the data recording was working. 

\begin{figure}[!ht]
    \centering
    \includegraphics[width=0.8\textwidth]{figures/map.png}
    \caption{Study operators view of the participants position and gaze direction.}
    \label{fig:map}
\end{figure}
% TODO: update grafik


\section{Procedure}




\subsection{Cybersickness}
After the user study not a single user reported any problems with cybersickness or quit the experiment even after a session of 40 minutes. Cybersickness was only a problem during the implementation testing where the detection algorithm was still in development.


\section{Results}


The distance of every teleport was recorded in both experiments. A 


In general the difference between the gestures is not pronounced. The quantitative differences are mainly not statistically significant and the opinions the participants expressed during the experiment and in the interview were sometimes just opposite to others. Some participants reported the opening and closing of the entire hand to form a fist as a confirmation step for the palm gesture to be very tiresome, while others specifically reported this to be the easiest part and that they could do this all day. 

As part of the final quantitative questionnaire after all experiments were conducted, the participants were asked to come up with ideas for other gesture systems, similar to the gesture elicitation study. It was to be expected that the answers would be biased by the study up to this point but this task was included because this study included more participants that now also had some experience with how the teleportation using hand-tracking feels like. As expected, the answers were largely based on the previous gestures but some interesting ideas could still be collected. Multiple participants came up with the idea of a one handed version of the triangle gesture system. Also one participant suggested to change the palm gesture systems target direction so the targeting hand would be in a more comfortable position and the direction vector would not change when the hand is closed to confirm the teleport location. The change they proposed would have the hand in an orientation where the thumb would face towards the user with the hand edge pointing in the teleport direction. Both ideas seem like a promising implementation of the participants feedback and could be interesting to investigate in future research.
