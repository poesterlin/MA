
\section{Gesture Elicitation Study}
To be able to evaluate the intuitiveness of the implemented gestures, a Gesture Elicitation Study was performed. The study was conducted with 5 participants, aged ... The participants where instructed to come up with teleportation techniques and to explain them in detail. To understand the limitations of the hand-tracking they where wearing the Oculus Quest device with hand-tracking enabled. This way it is easy to tell which gesture is tracking well and what is not detected by the tracking cameras. A simple environment with only a plain on the bottom, the skybox and the users tracked virtual hands was used. The environment was created so that the focus lies on the hands and the gestures. During the study, the headset was connected to the websocket relay application. Using a simple web interface, the study operator could connect to the websocket as well and send commands to the headset. Pressing "save" button on the web application records the current gesture. It is send to the main server application to allow the detailed analysis of gesture in the Visualizer after the study. This was done because it would not have been realistic to record everything the participant does with their hands. The websocket then allows the communication with the headset with very little delay, so that important gestures can be saved when the participant is trying something out. The Visualizer also allowed the operator to check if the recording of the gesture data worked during the study. The gesture data was backed up after each participants run. The conversation between the operator and the participants was also recorded on video for analysis.
