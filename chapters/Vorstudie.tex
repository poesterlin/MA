
\chapter{Gesture Elicitation Study}
To be able to evaluate the intuitiveness of the implemented gestures, a Gesture Elicitation Study was performed. % explain and cite


\section{Participants}
The study was conducted with 6 participants. The participants came from the Universities Human-Computer Interaction working group. All participants where male, students (2) or PHD students (4), aged $28.5$ on average (min: $26$, max: $35$) and volunteered to take part. Before the main part of the study, the participants were asked to rate how much experience with virtual reality they have on a scale from 1 to 5, 1 being not at all and 5 being professional experience. The resulting average experience is $4.33$ out of 5, with a median of $4.5$. Four out of the six participants also reported to have previous experience with VR applications that use hand-tracking. With two participants citing the Oculus Quest hand-tracking demo application and two participants citing a custom application using the leap motion tracking device. % cite
This means a third of participants already have experienced gesture based teleportation before. 


\section{Study Setup}
The participants where instructed to come up with teleportation techniques and to explain them in detail. One and two handed gesture systems are allowed. To understand the limitations of the hand-tracking the subjects where wearing the Oculus Quest device with hand-tracking enabled. This way it is easy to tell which gesture is tracking well and what is not detected by the tracking cameras. A simple environment with only a plain on the bottom, the skybox and the users tracked virtual hands was used. The environment was created so that the focus lies on the hands and the gestures. During the study, the headset was connected to the websocket relay application. Using a simple web interface, the study operator could connect to the websocket as well and send commands to the headset. Pressing "save" button on the web application records the current gesture. It is send to the main server application to allow the detailed analysis of gesture in the Visualizer after the study. This was done because it would not have been realistic to record everything the participant does with their hands. The websocket then allows the communication with the headset with very little delay, so that important gestures can be saved when the participant is trying something out. The Visualizer also allowed the operator to check if the recording of the gesture data worked during the study. The gesture data was backed up after each participants run. The conversation between the operator and the participants was also recorded on video for analysis. The sessions lasted on average about six minutes.


\section{Results}
28 different gesture systems were collected from the video and gesture snapshot information, with five systems appearing twice. Nine gesture systems are not usable for the comparison since they are either not strictly teleportation systems or because they can not be compared to other teleportation methods like a system using a minimap ar a type of proxy system. Out of the usable 19 systems, six use a bimanual approach while 13 use one hand. This trend to one handed gesture systems was also explicitly called for by two participants that expressed some possible downsides of bimanual systems. The participants reported more physical effort and not being able to carry something in one hand while teleporting as a disadvantage.

The usable gestures can be categorized into three large groups of similar gesture systems. 

The largest group of systems are all using at least one hand with the index finger extended to use as a pointing device. In total this method was proposed eight times. One gesture is also using and extended middle finger to make the gesture more distinct (\ref{fig:index2}) but the targeting system otherwise works the same. Six times the system was proposed to have a confirmation step before the teleport is actually performed. According to the participants this should work by using the a "finger gun" type gesture where the thumb is first extended and is then tapped against the base of the index finger to confirm the teleport location as seen in \ref{fig:index}. One user also proposed to gesture an "air tap" with the extended index finger. However, he expressed some concern about the accuracy, since moving the finger to confirm could impact the target selection. Twice the second hand was used as a confirmation step but had no influence on the targeting system using the pointing hand.

\begin{figure}[hbt!]
    \centering
    \includegraphics[width=\textwidth/2]{figures/index.jpg}
    \caption{Index finger pointing gesture with two stages.}
    \label{fig:index}
\end{figure}

\begin{figure}[hbt!]
    \centering
    \includegraphics[width=\textwidth/5]{figures/double index.jpg}
    \caption{Pointing gesture with index and middle finger extended.}
    \label{fig:index2}
\end{figure}

The second biggest group of gestures was named six times. All gestures are using a single, open hand with some kind of confirmation step. Four times closing the hand to a fist was proposed as a confirmation step, with the others quickly tapping the index finger and the thumb together to select the target. Another difference between the systems is also the direction the open hand is pointing towards. Four times a ray would start as the normal vector of the palm of the hand, once out of the middle finger. One other system proposed to have the palm upwards, with an arc used as the targeting visualization. The arc would curve in the direction of the middle finger and could be manipulated by changing the hight of the hand. Holding the hand up high makes the arc go further and would therefore allow to teleport a larger distance.

\begin{figure}[hbt!]
    \centering
    \includegraphics[width=\textwidth/2]{figures/palm.jpg}
    \caption{Palm gesture with two stages.}
    \label{fig:index2}
\end{figure}

A third group is named five times and is using only bimanual systems that use a targeting system where the selection vector is produced by some kind of "rangefinder", the user is looking through. This could be a triangle formed by touching both index fingers and both thumbs together. This was proposed four times. Another proposed option is a "diamond" form formed by touching all fingers to their equivalent finger of the other hand. This forms a hole that can be used to look through. This was proposed twice. All but one systems also include a confirmation step that is performed by closing other fingers to a half fist or pinching the hole together.

Other honorable mentions are:
\begin{itemize}
    \item Wrist mounted laser pointer
    \item Pointing using a thumb
    \item OK-Gesture with using the middle finger to point, confirmed when opening the sign
    \item Throwing a teleport ball to a target
    \item Drawing a circle in the air that will become the new viewport
\end{itemize}

They were all mentioned only once but could also be interesting to investigate.

\section{Discussion}
The results are very promising for the usability study and for use in all kinds of applications since all the methods implemented to test were also intuitively proposed by the test subjects. Also the change to add a confirmation step is well supported by the participants. Only 3 out of 19 gesture systems proposed by the participants did not include an explicit explanation for a confirmation step. 
