\chapter{Outlook}

The implemented systems are working well and can be easily adapted for more types of gestures with fast development using the debugging and visualization systems. However there are also some limitations. The gesture detection system is fixed to keyframe like gesture definitions. That means it requires an approximate match of all joints of a hand to be recognized as a new state. That means it would be harder to build systems that calculate a value based on the current hand position. This would require higher identification thresholds or the use of many in-between states, to be able to track for example the distance between a half pinched index finger and thumb. This was not required right now but could be a requirement in the future to be able to give user feedback if the detection threshold for a gesture is almost met. 

A limitation of the development process is that there always needs to be an instance of the backend running and the headset requires an internet connection for the initial download of gestures. However, this could easily be fixed by caching the last gesture set on the local filesystem to be able to have a fallback version. The current setup does require the connection for other reasons anyway though so this was not done.