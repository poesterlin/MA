\chapter{Outlook}

The implemented systems are working well and can be easily adapted for more types of gestures with fast development using the debugging and visualization systems. To be able to conduct the study, a way to switch between the environments has to be implemented and tested well. This is important to get right since this would have to include some way of informing the user how the teleportation gestures work. This needs to be reliable otherwise it could frustrate the user or require the conductor of the study to step in and help.  

After the future study, the collected information has to be analysed so there needs to be a way to generate various diagrams from the data. The first steps for this namely the map visualization is already implemented but this needs to be expanded significantly to show multiple runs with more detailed data and options. So far the project is on a good track to produce results that could be used to help future developers of applications using hand-tracking make informed decisions concerning locomotion.