\section{Motion Sickness}\label{motion-sickness}

The phenomenon known today as motion sickness predates modern technology
by millennia and even Hippocrates wrote about it. The word ``nausea'',
the main symptom of motion sickness is also derived from the Greek word
for ship ``naus''. \cite{Golding} It is a very
unpleasant feeling and was even used as a legal punishment in the 19th
Century \cite{Reason}. The source of motion sickness is
well understood because there is a lot research from military agencies
since they had to transport a lot of soldiers by ship and needed them to
be healthy when they reached their destination
\cite{Johnson}. After the invention of the first flight
simulator, the term simulator sickness was coined. Simulator sickness is
a form of motion sickness that can occur when training pilots in
simulator \cite{Johnson}. With the rise of head mounted
displays and VR technology, motion sickness got another subcategory:
virtual reality sickness, also known as cybersickness. It is distinct
from simulator sickness in that the symptoms are not as much related to
the Oculomotor system (for example eyestrain, blurred vision, etc) but
rather disorienting. The severity was found to be about 3 times greater
than simulator sickness. This might be because the VR systems are more
immersive than a flight simulator that relies on traditional displays
\cite{Stanney}.\\Because cybersickness and simulator
sickness are different, there needs to be a separate evaluation method
for the two phenomena. According to Stone et al. this is because the
weighted scale of the simulator sickness questionnaire does not
translate well to the VR environment. This is improved with a new
weighting scale proposed by Stone et al. but a new questionnaire
specifically created for cybersickness generates the best results.
However the simulator sickness questionnaire is often used in reached
anyway. This only has the benefit that the results are easy to compare
but there are significant tradeoffs in regards to accuracy.
\cite{Stone}

\subsection{Cybersickness}\label{cybersickness}

As discussed before, cybersickness is a form of motion sickness. It can
result in a range of symptoms including nausea, vomiting,
disorientation, headaches and eye strain
\cite{LaViola}. This is a serious problem and needs to
be taken into account when developing VR systems.

% TODO: indent direct quote
The actual cause of cybersickness is not
known and the underlying physiological responses uncertain. The three
most prominent theories for the cause of cybersickness are poison
theory, postural instability theory and sensory conflict theory
\cite{Davis}.

\begin{itemize}
\itemsep1pt\parskip0pt\parsep0pt
\item
  Poison theory: survival mechanism that induces vomiting and nausea to
  remove poison from the body if the brain detects sensory input like
  hallucinations.\\
\item
  Postural instability theory: a loss of postural control causes
  sickness\\
\item
  Sensory conflict theory: symptoms are created if there is a conflict
  between the vestibular and visual senses. For example if a user is not
  moving but their avatar in VR is.
\end{itemize}

Unfortunately all three theories have low predictive power and fail to
explain some key aspects of cybersickness \cite{Davis}.

One way to reduce cybersickness is to break the optical flow the user
perceives using different kinds of techniques
\cite{Bhandari}. Optical flow is a phenomenon that
allows an observer to gather information about the motion of objects and
give the observer a sense of presence \cite{Gibson}.

\section{Locomotion in VR}\label{locomotion-in-vr}

The ability to move the players avatar in the virtual environment is
called locomotion. It is required for many applications or games that
take place on a larger scale. Without locomotion techniques VR would be
very inaccessible. A user would need a giant tracking space, which is
not possible in most homes and also the technology to track the users
headset would have to work over that amount of space. Locomotion is also
needed even if you could walk everywhere in your space, since it can
also be a convenience or accessibility feature in combination with a
seated experience. Getting the locomotion mechanics just right, will
also improve other interactions and user satisfaction in general. Moving
to a different virtual altitude would also not be possible without some
technique that can simulate the player walking up stairs. Otherwise
every user would have to have stairs in their tracked space. The
implementations of locomotion techniques can be very divers and so there
is a wide range of categorizations schemes.

Luca et al. collected 109 different locomotion techniques from academic
sources, social media and popular VR games. The result is a public
database of methods that can be filtered and compared. Some of the
methods are in a preliminary state and not fully implemented or
evaluated but in general the database is a great resource
\cite{Luca}. There are only 8 results for hand-tracking
are some of them are falsely categorized.

\subsection{Locomotion using
controllers}\label{locomotion-using-controllers}

Boletsis et al. \cite{Boletsis} categorized the four
prevalent locomotion techniques into:

\begin{itemize}
\itemsep1pt\parskip0pt\parsep0pt
\item
  Room-scale-based: uses physical movement, translates movement from the
  read world one to one into VR. continuos movement, unlimited range.\\
\item
  Motion-based: uses physical movement for example swinging arms or
  walking in place. Continuos movement, unlimited range\\
\item
  Controller-based: The uses controller input like a joystick to move.
  Continuos movement, unlimited range.\\
\item
  Teleportation-based: the viewpoint is instantly moved to a new
  predefined location. Non-continuos, unlimited range
\end{itemize}

The room-scale-based techniques are limited to the size of the tracking
space so they are not flexible enough for a general use case. If the
task allows it, it should however be the preferred\\locomotion technique
because it results in the best immersion while also keeping
cybersickness to a minimum.

Controller-based techniques could be adapted into a hand-tracking
environment, the continuos nature of the movement, without some kind of
a physical representation of the movement performed by user however
would make it prone to create motion sickness.

Motion-based techniques are very dynamic because of their physical
nature. That makes them hard to track with current hand-tracking
technology. They also made users tired the fastest.
\cite{Boletsis}

For those reasons the scope of this work will focus on only
teleportation-based techniques to give results that as widely useable as
possible and accessible to the most amount of people without inducing
cybersickness.

\subsubsection{Teleportation-based
locomotion}\label{teleportation-based-locomotion}

Teleportation is one of the most used locomotion techniques. Each
implementation can be slightly different in the way it is integrated
into the virtual environment but the core mechanic allows the user to
move the viewpoint to points on the map using some ray casting system.
There is usually a limit to the teleport distance and sometimes the user
is only allowed to select predefined locations as targets. The technique
is categorized as non-continuos movement with unlimited range. Since not
only the xy coordinates of a target location can be chosen but also
points at different altitudes, the method has 3 degrees of freedom.

The research from Clifton et al. shows that on average teleportation
causes less cybersickness than continuos navigation, however, there were
some people (38\%) that had the opposite reaction.
Clifton et al. conclude that there should always be multiple locomotion
methods to choose from. The researchers theorize that the cause of the
cybersickness might be that some people are more sensitive to the
immediate displacement used by teleportation-based locomotion. If the
participants experienced the virtual environment while sitting or
standing did not make a cause a meaningful change in the reported
effects, only that there seams to be slightly worse cybersickness from a
seated experience. \cite{Clifton}

However, teleportation is also not the perfect solution:

\paragraph{Problems:}\label{problems}

The lack of optical flow between locations is great to minimize
cybersickness, but it also introduces limitations:

Path integration, a process where the brain updates the current position
continuously using information from different senses. Visual, vestibular
and proprioceptive sensory input are continuously integrated and create
a rough estimate of the distance traveled.
\cite{Bhandari}

Bhandari et al. found that this can be improved by allowing some optical
flow between teleport points without creating higher levels of
cybersickness. This can be achieved by using a technique the researchers
call ``Dash''. It translates the user at a constant velocity from the
current point to the target. This creates a short transition period that
takes a maximum of 1.1 seconds over a maximum distance of 11 meters in a
virtual environment at real world scale. Previous research indicates
that the duration and speed do not impact the resulting effect a lot
\cite{Bowman} but these values where picked after some
internal testing. Using this technique users where able to move back to
a starting point after teleporting away much more accurately without any
landmark or context clues. \cite{Bhandari}

An interesting problem with teleportation-based locomotion is that is
was found to be the least immersive technique out of the four types
categorized by Boletsis et al. \cite{Boletsis}. This
could improve using hand-tracking because the users can use their own
hands and that might be more immersive for some people. (maybe included
in hypotheses)

\subsection{Immersion}\label{immersion}

\subsection{Locomotion using
hand-tracking}\label{locomotion-using-hand-tracking}

When filtering the locomotion database from Luca et al. for techniques
that are categorized to use hand-tracking only 8 results are presented.
The results that are relevant for this paper are:

\begin{itemize}
\item
  Hand Close: a gesture for continuos movement. Seams to have problems
  with cybersickness (25\% of study participants had to
  abort the 90 minute test). \cite{Huang}
\item
  Finger Run: two-handed gesture for continuos movement, the only
  resource is a video on a game development account that mostly posts fun
  prototyping ideas. \cite{Beauchamp}
\item
  Walking stick: a gesture instantiates a walking stick, that can be
  used to move the user in relation to the sticks touch point on the
  ground. This is only a preliminary database entry though and this is
  not a great solution for big distances or different altitudes.
\end{itemize}

The categorizations seam to have some issues and so going through all
109 techniques, a small list of more gestures can be found:

Technologies that might work using gestures:

\begin{itemize}
\itemsep1pt\parskip0pt\parsep0pt
\item
  World in miniature: manipulate player position on a miniature version of the environment
\item
  Cloudstep: miniature teleportation steps using a joystick
\item
  Dash Pointing: fast, continuos step in the direction of the controller with limited field of view
\end{itemize}

Problems:

\begin{itemize}
\item
  limited tracking space for hands: Controllers can still detect button
  presses and orientation changes if they are out of the tracking
  region, like behind the body. The hand-tracking system has no idea
  what the users hands are doing if they are out of view or obscuring
  each other.\\
\item
  limited accuracy: Hand-tracking is much less accurate with current
  technology. This might improve in the future though.\\
\item
  Hand-to-hand interaction is confusing the hand-tracking system\\
\end{itemize}

\section{Gestures}\label{gestures}

\subsection{Gesture Detection}\label{gesture-detection}

The most accessible hand-tracking hardware for virtual reality today is
the Oculus Quest 2 VR headset. It has hand-tracking build in, which can
be accessed using the Oculus SDK in Unity. The exact technology the
Oculus Quest device uses is not known. However it can be speculated that
Oculus is using a variation of the research done by Han et al. The
Oculus Quest only has monochrome cameras and only a very limited amount
of processing power, which correlates with the limitations addressed by
Han et al. \cite{Han}. Like the Oculus Quest, the
unnamed headset in the paper has 4 cameras. They are positioned on the
headsets corners, all facing in slightly different directions to cover a
120° area in front and below the headset. The areas the individual
cameras can see overlap in some places so that in the most important
areas the hands are visible for at least 2 up to 4 cameras.\\The
tracking system works by first detecting the hands in the camera streams
using a CNN architecture optimized for efficiency named DetNet. The
network can simultaneously localize and classify the hands which
combines two steps into one. Next, a network called KeyNet takes the the
cropped camera images and outputs 21 coordinates that match key points
of a human hand. The network architecture is optimized to reduce jitter
and consistency when moving between overlapping camera areas. This is
done using extrapolated points as an additional network input. The
output is suitable for basic interactions and certainty impressive for
real time processing on mobile hardware but there are definite
limitations. Since Hand-to-hand occlusions and interactions were not
part of the training set, the output has problems with complicated
poses.

\subsection{Future Technology}\label{future-technology}

Current technology has some major limitations when tracking hands.
Hand-to-hand interactions, self-contact and occlusion come to mind.
Smith et al. created a system that takes in high resolution data from
124 cameras capturing uniformly lit hands. The images are turned into a
mesh so detailed its basically indistinguishable from the original
hands. The pose of two hands and even the texture is captured without
any artifacts. This process is very complex and it takes minutes to
render per frame so it will take some optimization until detailed hand
models like this could be included in a VR environment. However, this is
the direction that technology will go, even if it will take some time.
That means that even if hand-to-hand interaction is not able to be
reliably tracked currently, it could be in the future.

\subsection{Intuitive and comfortable gestures}\label{ergonomics}
For the design of a hand gesture vocabulary, the first consideration might be how well the gesture can be identified. However, gestures that are easily recognized may not be intuitive and easy to perform by the user. According to Stern at al. there are three factors that all should be maximized when choosing a gesture: intuitiveness, comfort and recognition accuracy. \cite{Stern2006}

Stern et al. define intuitiveness as the cognitive association between a command or intent and its physical gestural expression. Experiments that trying to find universal gestures for a virtual reality environment were also done by Pereira et al. \cite{Pereira2015}. The in the experiment 34 participants were asked to come up with gestures for common HCI tasks. This resulted in over 1300 different gestures for 34 tasks. However, only 84 gestures were chosen by 3 or more subjects and some of them for different tasks. Expecting the user to intuitively guess the same gesture that the developers of the virtual environment intended therefore seams unrealistic. Constructing a gesture that requires no introduction or tutorial is therefore also not the goal of this paper.

The research of Ardito et al. also supports this. They compare trying to design universal gestures to problems that arose with visual languages in the past. They predict that trying to use the same gesture vocabulary over a range of applications and devices is likely to fail. \cite{Ardito2014}


As a measure for comfort, Kölsch et al. propose a method that distinguishes between a pose of some joint or body part and the user compensating using other joints to relieve strain. A compensation for a pose that requires the users head to be tilted onto its side might be to keep the head upright relative to the body but bend the torso to the side. To find the optimal comfort zone of a posture, Kölsch et al. look how users compensate or if they execute the pose without compensating. \cite{Koelsch}

There are also other methods that rely on biomechanical models and physics simulations to measure how comfortable a gesture is but they are prone to errors \cite{Stern2006}.

Another source of knowledge to source information from regarding gestures are sign language interpreters. They might spend more than 20 h per week signing, however, they also often suffer from hand pain and fatigue while gesturing for long sessions. This would also be a risk for prolonged sessions in virtual reality that is using hand-tracking gestures. Rempel et al. conducted a study with experienced sign language interpreters to find the discomfort level associated with signing selected hand postures, motions and characters \cite{Rempel2014}. 

Rempel et al. collected findings and summarized:
\begin{itemize}
  \item Hands should be near the midline of the body and not further apart than the shoulders. They should be near the height of the elbows and close to the lower chest, so that the shoulder muscles are not tense. 
  \item Sustained elbow flexion of more than 90 degrees should be avoided.
  \item The most comfortable postures were between neutral and 45 degrees of pronation (palms rotated toward ground). Repeated full rotations to palm up (supination) or palm down (pronation) should be avoided.
  \item Gestures that included movement of the elbow or shoulder were more comfortable than gestures with just finger movements.
  \item The more comfortable gesture motions were up-down hand movements performed with motion at the elbows or side-to-side hand movements with motion at the shoulders. 
  \item the most comfortable hand gestures were with the wrists straight and the fingers slightly flexed or in a loose fist.
  \item The least comfortable gestures were those involving wrist flexion; discordant adjacent fingers; or fingers extended.
  \item The position of the thumb did not appear to have a large influence on comfort.
\end{itemize}



