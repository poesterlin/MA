
\chapter{Introduction}
The world of virtual reality is getting increasingly diverse. There are all kinds of tools and games build pushing the boundary of what is possible to create. The hardware is also getting more powerful and easier to use. Standalone headsets that do not require a powerful computer linked over a cable are cheaper and can even be more immersive since the user does not has to keep track of the cable. The next step on the path of simplifying the experience is to make it possible to use VR without controllers. They will still have their place for applications that require very acuate tracking, lots of different types of inputs or where it makes sense to hold something like a gun or paintbrush anyway. Many experiences do not require this though. Especially for applications and tools used to collaborate, enabling users to enhance their communication with gestures might be helpful. Use-cases like this are where hand-tracking will allow the implementation of immersive, natural interfaces. This can allow anybody to just put the headset on and just start interacting with the virtual world without first being told how to grab something or how to activate a button. However, the interactions do have limitations as stated before and are a challenge to get right. Users are first going to apply the mental model build during decades of experience using their hands in the real world. Differences between the real and the virtual world might therefore make the experience less immersive than if the user would be using a controller where there is no previous knowledge. This raises the question of what to do if the virtual world allows the user to do more with their hands than in the real world. 

This work builds the basis to be able to investigate how locomotion or more specifically teleportation should work using hand-tracking. Teleportation is the most popular form of locomotion in VR. When using controllers, locomotion is usually controlled using a joystick or button and even though there are slight differences between each game, you can usually figure out quickly how to do control it. Without controllers, there is no established way to control teleportation. The user can also not easily hit all the buttons until some feedback can lead them in the right direction. VR applications based on hand-tracking will therefore still have a learning curve for new users. If there is a standard way to use teleportation established it should be based on research though.

% goal of the thesis

% maybe structure





