
\chapter{Introduction}
The world of virtual reality (VR) is getting increasingly diverse. There are all kinds of tools and games build pushing the boundary of what is possible to create with current technologies. The computing hardware is also getting more powerful and easier to use. Standalone headsets that do not require a powerful computer linked over a cable are cheaper and can even be more immersive since the user does not has to keep track of the cable. The next step on the path of simplifying the experience could be to make it possible to use VR without controllers. They will still have their place for applications that require very acuate tracking, lots of different types of inputs or where it makes sense to hold something like a gun or paintbrush anyway. Many experiences do not require this though. Especially for applications and tools used to collaborate, enabling users to enhance their communication with gestures might be helpful. Use-cases like this are where hand-tracking will allow the implementation of immersive, natural interfaces. This can allow anybody to just put the headset on and just start interacting with the virtual world without first being told how to grab something or how to activate a button as shown in \ref{fig:example}. However, the interactions do have limitations in terms of accuracy and their number of inputs as stated before and are a challenge to get right. Users are first going to apply the mental model build during decades of experience using their hands in the real world. Differences between the real and the virtual world might therefore make the experience less immersive than if the user would be using a controller where there is no previous knowledge. This raises the question of what to do if the virtual world allows the user to do more with their hands than in the real world. 

\begin{figure}[!ht]
    \centering
    \includegraphics[width=\textwidth]{figures/examplegestures.png}
    \caption{Example of users interacting with objects in VR using hand-tracking. \cite{Han}}
    \label{fig:example}
\end{figure}


One simple way where VR technology can expand on what is possible in the real world is scale. The area a user is physically able to explore and walk around in comfortably is limited. For most casual users of VR that might be the size of their living room. In VR this is not a limitation. There a user is able to fly, teleport or use any number of so called locomotion techniques to increase their level of comfort and allow them to go wherever they want. While not completely understood, the disconnect between the movement in the real word and in VR is likely one source of motion sickness. This makes locomotion a necessary mechanic to implement in large scale VR experiences but also a very divisive one since it can impact the users immersion and level of comfort a lot. VR experiences only offering continuos locomotion can, for some users that are more prone to get motion sick, only be usable for a few minutes, only with a limited field of view or not useable at all. On the other hand users that are used to continuos navigation, do not get motion sick as quickly or even at all report being frustrated by having to use another, less immersive locomotion mechanic that might get in the way a lot more. So while teleportation based locomotion is not needed for everyone it allows most people to at least use large scale VR experiences comfortably without getting sick. This makes it the first and sometimes only mechanic that is implemented and so it should be as well understood and implemented as possible. Especially when in combination with hand-tracking and gesture based teleportation there is only very limited research done so far.

This work is expanding the research previously done on how locomotion or more specifically teleportation should work using hand-tracking. Teleportation is the most popular form of locomotion in VR. When using controllers, locomotion is usually controlled using a joystick or button and even though there are slight differences between each game, you can usually figure out quickly how to do control it. Without controllers, there is no established way to control teleportation. The user can also not easily hit all the buttons until some feedback can lead them in the right direction. VR applications based on hand-tracking will therefore still have a learning curve for new users. If there is a standard way to use teleportation established it should be based on research though. 

% goal of the thesis

% maybe structure

% bilder



