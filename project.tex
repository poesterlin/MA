% HCI Group Konstanz Seminar Template
% Version 1.1
%
% ATTENTION: This template was designed to fit the most common requirements for different types of reports (e.g., Seminar to the Project or Project Report). However, if you feel the need to add an extra component or layout feature please talk to your advisor. Please do not change anything on this template unless it is explicitly allowed or agreed with your advisor. However, it is allowed to add packages that do not alter the design/layout of this template.
%

\documentclass[10pt, paper=a4, parskip, oneside]{scrreprt}
\usepackage[T1]{fontenc}
\usepackage[utf8]{inputenc}
\usepackage{hciknseminar}
\usepackage[ngerman, english]{babel} % Spell checking (Using ngerman for german spell checking). The last language is the standard language for the document.
\usepackage{enumitem}
\usepackage{booktabs}
\addbibresource{bibliography.bib}

% =========== Title page ============
\title{Evaluating Gesture Based Teleportation Techniques For Immersive Virtual Environments} % Your title
\type{Master Project Report} % Seminar to the Bachelor/Master Project | Bachelor/Master Project Report
\author{Philip Oesterlin} % Your name
\studentno{01/993546} % Your student number
\group{Human-Computer Interaction} % Do not change this
\department{Computer and Information Science} % Do not change this
\advisor{Jonathan Wieland} % Your advisor
\reviewer{Prof. Dr. Harald Reiterer} % Do not change this
\date{\the\day .~\monthname~\the\year}

% ============= Abstract =============
\abstract{
    Advances in hand-tracking for VR applications allow users to experience virtual environments without controllers. There are already successful implementations of this however many are still stationary experiences. Using hand-tracking requires new locomotion techniques since there are no joysticks or buttons to press to move the users' avatar and there are no established solutions for this problem yet. This report describes the implementation of a flexible teleportation system using hand-tracking. The gesture detection is based on recorded gesture states that can be compared to the current joint positions.
    The project was conducted with the goal of preparing a study environment to compare different gesture-based teleportation systems that were found during a literature review as well as in commercial VR games. To check that the chosen gestures are intuitive, a small gesture elicitation study was performed. 
    The main part of this project is a VR environment, build with Unity and a server connected to the headset for various logging, visualization and debugging tasks.
}

% ============= Document =============
\begin{document}
% Title page
\maketitle
% Abstract
\makeabstract
% Table of contents
\tableofcontents
\addcontentsline{toc}{chapter}{Contents} % Add contents to table of contents
% List of Figures
\listoffigures
\addcontentsline{toc}{chapter}{\listfigurename} % Add list of figures to table of contents
% List of Tables
\listoftables
\addcontentsline{toc}{chapter}{\listtablename} % Add list of tables to table of contents
% Prepare chapters
\clearpage
\setcounter{romanPages}{\value{page}} % Update variable for roman pages
\pagenumbering{arabic} % Turn on page numbering again

% ============= Chapters =============


\chapter{Introduction}
The world of virtual reality (VR) is getting increasingly diverse. All kinds of tools and games are built, pushing the boundary of what is possible to create with current technologies. The computing hardware is also getting more powerful and more consumer-friendly. Standalone headsets that do not require a powerful computer linked over a cable are cheaper and can even be more immersive since the user does not has to keep track of the cable. The next step on the path of simplifying the experience could be to make it possible to use VR without controllers. They will still have their place for applications that require very acuate tracking, lots of different types of inputs or where it makes sense to hold something like a gun or paintbrush. Many experiences do not require this, however. Especially for applications and tools used to collaborate, enabling users to enhance their communication with gestures might be helpful. Use-cases like this are where hand-tracking will allow the implementation of immersive, natural interfaces. This can allow anybody to just put the headset on and start interacting with the virtual world without first being told how to grab something or how to activate a button, as shown in figure \ref{fig:example}. However, as stated before, the interactions do have limitations in terms of accuracy and their number of inputs and are a challenge to get right. Users will first apply the mental model built during decades of experience using their hands in the real world. Differences between the real and the virtual world might therefore make the experience less immersive than if the user were using a controller where there is no previous knowledge. This raises the question of what to do if the virtual world allows users to do more with their hands than in the real world. 

\begin{figure}[!ht]
    \centering
    \includegraphics[width=\textwidth]{figures/examplegestures.png}
    \caption{Example of users interacting with objects in VR using hand-tracking. \cite{Han}}
    \label{fig:example}
\end{figure}


One simple way where VR technology can expand on what is possible in the real world is scale. The area a user is physically able to explore and walk around comfortably is limited. For most casual users of VR, that might be the size of their living room. In VR, this is not a limitation. There, a user can be given the ability to fly, teleport or use any number of so-called locomotion techniques to increase their level of comfort and allow them to go wherever they want. While not completely understood, the disconnect between the real world's movement and VR is likely one source of motion sickness. This makes locomotion a necessary mechanic to implement in large scale VR experiences and a very divisive one since it can impact the users' immersion and level of comfort a lot. VR experiences only offering continuous locomotion can, for some users that are more prone to get motion sick, only be usable for a few minutes, only with a limited field of view or not useable at all. On the other hand, users that are used to continuous navigation do not get motion sick as quickly or even at all report being frustrated by having to use another, less immersive locomotion mechanic that might get in the way a lot more. So while teleportation based locomotion is not needed for everyone, it allows most people to at least use large scale VR experiences comfortably without getting sick. This makes it the first and sometimes only mechanic that is implemented, and so it should be as well understood and implemented as possible. Especially when in combination with hand-tracking and gesture-based teleportation, there is only minimal research done so far.

This work expands the previously done research on how locomotion or, more specifically, teleportation should work using hand-tracking. Teleportation is the most popular form of locomotion in VR. When using controllers, locomotion is usually controlled using a joystick or button, and even though there are slight differences between each game, you can usually figure out quickly how to control it. Without controllers, there is no conventional way to control teleportation. The user can also not easily hit all the buttons until some feedback can lead them in the right direction. VR applications based on hand-tracking will, therefore, still have a learning curve for new users. If there is a standard way to use teleportation established, it should be based on research. 

% goal of the thesis

% maybe structure

% bilder





% - How do the results of the seminar work inform the project?
% - Which requirements need to be addressed from a technical point of view to realize the concept that was derived in the seminar work (interaction concept/study setting)?
\chapter{Results from the Literature Research}

There is not much research focused on alternative locomotion techniques in VR that do not require controllers. It is a new and experimental field of study.

\section{Tracking Technology}
Locomotion systems using hand-tracking traditionally required external tracking devices like a leap motion sensor mounted to the VR headset developed by the company Ultraleap \cite{Ultraleap}.
This works for a prototype but can not be expected to be something an end-user is able to set up. The studies conducted using the leap motion sensor also reported users complaining about the sensors field of view like in the study conducted by Schäfer et al. \cite{Schafer2021}.
The sensor has a 140° field of view and therefore can see the environment in front of the headset well, however it is not able to track hands if they are held below the headset or next to the users' body, where it would be more natural and ergonomic. A much more usable implementation of hand-tracking is implemented by the Oculus Quest headsets. It does not require any cables or additional hardware. The headset is internally processing the images of up to 4 cameras that are able to detect the users' hands using a trained neural network developed by Han et al \cite{Han}.
Even though the cameras only produce black and white images and sometimes only one camera is able to see the hand, the tracking works remarkably well. There are however major limitations since the detection network is not trained to detect hand-to-hand interactions and therefore cuts out entirely until the user separates their hands again. The tracking is also sometimes inaccurate if some fingers are obstructed by others. In most cases this is not very visible to the user since the fingers are also obstructed from their perspective, it is however visible in the tracking information recorded by the headset and has to be taken into account. 

\section{Requirements for Teleportation Systems}
Enabling simulated movement in virtual environments is not a new problem. The techniques that allow the user to move their avatars viewpoint can be implemented in a lot of different ways that all can have different effects on different people. Therefore they require special attention in regards to many different factors. Otherwise moving can result in motion sickness, a low level of immersions, disorientation and an overall frustrating user experience. To produce a good VR experience, it is important to understand the users needs when it comes to locomotion and motion sickness. 

Motion sickness or more accurately cybersickness is a phenomenon not completely understood. It can be a problem in VR like it is when riding the train backwards, reading a book in a moving car and even for pilots in airplane simulators. VR should be fun and cause as little cybersickness as possible. To reduce the effects there are a couple of techniques. First, the application should perform well and run with a consistently high framerate. Performance is a key consideration when developing for VR. Modern hardware makes this less and less of a problem and it is possible to run even highly detailed applications with a high framerate. However, cybersickness will sill come up for some people if there is optical flow when moving in VR while standing still in real life. Optical flow is the phenomenon that allows the brain to gather information about the motion of objects the give the user a sense of presence. This effect produces an immersive VR experience if it does not stand in conflict with the sensors the human body has that are not tricked by a VR headset. Optical flow during movement in VR is especially a problem in the peripheral vision. To reduce this some applications allow the user to enable a virtual helmet that reduces the field of view. A good example of this would be the game Rollercoaster Simulator. A rollercoaster ride moves the player continuously in VR while they are standing still in real life. This can make even experienced VR users motion sick, so the implementation of the helmet makes a lot of sense. 

The most popular way to implement locomotion is to allow the user to teleport in the virtual environment. A teleportation jump instantaneously transports the avatars viewpoint to a new locomotion without any transition. This way the effects of motion sickness are reduced since there is no optical flow. No optical flow is a good way to reduce cybersickness, however it also has an effect on the immersiveness of the experience. Some users have a harder time experiencing the environment fully since the optical flow is also a major factor controlling the level of immersion. This is not ideal however it is not as sirius of an effect as some users not able to use the application at all because of the cybersickness symptoms. 

In traditional controller-based applications the focus lied on the implementation itself and activating the locomotion only required a button press. With hand-tracking, there is another human factors to consider, namely the ergonomics of repeating some action over potentially multiple hours. Ergonomics can be something different from person to person, there are some key rules to follow though. This includes allowing the hands to held at a natural hight, not requiring the elbow to be flexed to more than 90 degrees over a long time, not requiring hands to turn more than 45 degrees from the position of the palm facing the ground and allowing movement to mostly come out of the elbows and shoulders. 


\section{Gestures}
The literature has come up with a handful of gestures and ways to study them in a laboratory environment. The games industry however has implemented some fully usable systems for actual users. Both types of sources were taken into account to produce this list of comparable gestures:

\begin{itemize}
    \item One-Handed Palm Gesture by Schäfer et al. \cite{Schafer2021}:
    Researchers propose a teleportation gesture that is using a single, wide-open hand. The target location is selected using a ray starting as the normal vector of the palm of the hand. The teleportation is then executed after 1.5 seconds.

    \item One-Handed Index Gesture by Schäfer et al. \cite{Schafer2021}: 
    Researchers propose a teleportation gesture that is using one hand with the index finger extended to use as a pointing device. To select a point, a user can just point at it. The teleportation is then executed after 1.5 seconds.
    
    \item Two-Handed Palm Gesture by Schäfer et al. \cite{Schafer2021}: 
    Same targeting gesture as the One-Handed Index Gesture. The second hand is used to confirm the teleport by opening the hand.

    \item One-Handed Index Gesture by Schäfer et al. \cite{Schafer2021}: 
    Same targeting gesture as the One-Handed Index Gesture. The second hand is used to confirm the teleport by extending the index finger.

    \item Triangle Gesture found in the game Elixir by Magnopus \cite{Magnopus}:
    A game called Elixir which is a short hand-tracking demo for the Oculus Quest uses a bimanual gesture. Both hands form a triangle connecting both thumbs and index fingers. This can be used to target a point on the floor. To confirm the target location, the user pinches thumbs and index fingers together. 
    
    \item Pull Gesture found in the game Vacation Simulator by Owlchemy Labs \cite{VacSimOculus}:
    A game called Vacation Simulator has one of the best implementations of hand-tracking found in games so far. The locomotion is using a one-handed pulling gesture. In the game, this is only used for low-resolution target selection.

    \item Palm-steering Gesture by Caggianese et al. \cite{Caggianese}:
    This method is using the normal vector of the open palm to set a target. Once a target has been found it is locked as soon as the user closes their hand to a fist. The movement is then started and the user will be moved to the target over a period of time.
    However, since this gesture is using continuous movement and not teleportation and therefore is not really comparable to the others.
    
    \item Index-steering Gesture by Caggianese et al. \cite{Caggianese}:
    To target a point in space using this gesture, a vector is created from the headsets position through the tip of the users extended index finger. To start the movement the user has to extend their thumb as a confirmation.
    However, this gesture is using continuous movement and not teleportation and therefore is not really comparable to the others.
\end{itemize}

After my own subjective testing using a prototype of the final application, the systems using a time delay were found to be disorienting. Because the teleportation can not be confirmed explicitly it is not transparent to the user when it is going to be executed. The locomotion systems of the first four gestures are also very similar and lead to very similar results in the previous testing. Also the similarity to the palm- and index-steering gestures proposed by Caggianese et al. would make studying the two methods separately kind of redundant. Because of this, it was decided to combine the one- and two-handed systems. This results in two one-handed systems that have a confirmation step to execute the teleportation. To get feedback on how intuitive this change was, a gesture elicitation study was performed. The pull gesture was not implemented since it is only used for low-resolution target selection in the previous implementation. The triangle gesture is not described in the literature so it was be replicated as closely as possible. There could however still be small differences in the way the implementation works which might not represent the original gesture exactly. After the selection the final three gestures are introduced in \ref{tbl:gestures}.

Index Gesture: 
\begin{figure}[!ht]
    \centering
    \includegraphics[width=0.4\textwidth]{figures/indexinfo.jpg}
    \caption{The two steps of the index gesture}
    \label{fig:indexInfo}
\end{figure}

Palm Gesture:
\begin{figure}[!ht]
    \centering
    \includegraphics[width=0.4\textwidth]{figures/palminfo.jpg}
    \caption{The two steps of the palm gesture}
    \label{fig:palmInfo}
\end{figure}

\begin{table}[h!]
    \begin{center}
    \begin{tabular}{c p{5cm} p{5cm}}
    \toprule
     Name & Steps & Description \\ 

   \cmidrule(r){1-1}\cmidrule(lr){2-2}\cmidrule(l){3-3}
   Index Gesture
    & 

    \raisebox{-\totalheight}{
        \includegraphics[width=0.3\textwidth, height=60mm]{figures/triangleinfo.jpg}
    }

    & 
    A triangle formed using both hands is used to select a target.
    \\ \bottomrule
    \end{tabular}
    \caption{Introduction to all implemented gestures.}
    \label{tbl:gestures}
    \end{center}
\end{table}

\subsection{Gesture Detection}
To detect if the current hand position is part of a gesture Schäfer et al. \cite{Schafer2021} propose an algorithm
that detects gestures based on the finger state and the palm direction. The finger state can only be curled or
stretched. The authors give the example of a fist gesture that is characterized by to have all fingers in a curled
state. For direction dependent gestures like a thumbs up gesture, the algorithm also takes the direction of the
palm into account. This seams to have some limitations for gestures like an okay sign where the thumb and index
finger are touching and are nether in a curled nor stretched state. However without the actual implementation that
Schäfer et al. developed its hard to know if the okay sign could be recognized or not.

\section{Testing Methodology}
The internal test game studios are using are not known, so the sources for the testing methodology are limited. In the literature, many studies rely on completion time and the number of errors during a task as a measure of a good locomotion system. Examples are Schäfer et al. \cite{Schafer2021} and Bozgeyikli et al \cite{bozgeyikli}.
This might cause some problems however since theoretically a system that can teleport right to the end of the environment would have a much better completion time than a system with a shorter teleport range even though this would not be a good system for the user. To combat this, it is necessary to have the same maximum teleport distance to compare systems. However, it would be better to not rely on the completion time as a measure and rather look for actual signs of user satisfaction. The number of errors seems like a good measure though since it can be very frustrating and disorienting to be teleported to the wrong location. 

The testing environments used for the studies are very diverse. Sometimes the prototype implementations are using a detailed environment so that it is very immersive for the test participant, like Caggianese et al. \cite{Caggianese} and other times it is created so that it is not distracting the player and therefore very clean and simple like Schäfer et al. \cite{Schafer2021}.  
Since there is no one-to-one comparison between results of the same locomotion system in a different environment, there is not an established, way to approach this.

The task the user is supposed to execute in the testing environment differs also depending on the study, however, mostly it is about teleporting to checkpoints along a corridor like Schäfer et al. \cite{Schafer2021} propose or in an open space.
This way the user is forced to hit points in the world accurately. This mechanic is easy to test however it is also not that similar to a game or a productivity application that an actual user would use. The need to hit an exact point and teleporting around just to get from one point to another without an objective is rarely found. Therefore adding an objective like avoiding an enemy or searching for something in the environment while teleporting would create a more realistic testing environment. This puts the teleportation itself in the background as it would be in an actual implementation of a teleportation system. However, if the system is not satisfying to use, the user will be able to notice the deficiencies since it is impeding the objective. An objective like this would make the task completion measure even more of a problem since it would not be the fault of the teleportation system if the user can not complete the objective on the first try and then takes longer. 



% - How was the project realized? (methodology)
% - Which previous work (e.g., related to interaction concepts or algorithms), and other available products (software, hardware, and other materials) can help to address the requirements to which extent? (comparative analysis and clear reasons for choices that were made to realize the project)

\chapter{Gesture Elicitation Study}
To be able to evaluate the intuitiveness of the selected gestures, a gesture elicitation study based on the research of Villarreal-Narvaez et al. \cite{elicitation} was performed. This is an important step since some of the gestures have been changed to be more comparable and to improve the user experience. Using the study, the changes were proven to be intuitive and necessary. 


\section{Participants}
The study was conducted with 6 participants. The participants came from the Universities Human-Computer Interaction working group. All participants were male, students (2) or PhD students (4), aged $28.5$ on average (min: $26$, max: $35$) and volunteered to take part. Before the main part of the study, the participants were asked to rate how much experience with virtual reality they have on a scale from 1 to 5, 1 being not at all and 5 being professional experience. The resulting average experience is $4.33$ out of 5, with a median of $4.5$. Four out of the six participants also reported having previous experience with VR applications that use hand-tracking. With two participants citing the Oculus Quest hand-tracking demo application and two participants citing a custom application using the leap motion tracking device.
This means a third of participants already have experienced gesture-based teleportation before. 


\section{Study Setup}
The participants were instructed to come up with teleportation techniques and to explain them in detail. One and two-handed gesture systems are allowed. To understand the limitations of hand-tracking the subjects were wearing the Oculus Quest device with hand-tracking enabled. This way it is easy to tell which gesture is tracking well and what is not detected by the tracking cameras. A simple environment with only a plain on the bottom, the skybox and the users tracked virtual hands was used. The environment was created so that the focus lies on the hands and the gestures. During the study, the headset was connected to the WebSocket relay application. Using a simple web interface, the study operator could connect to the WebSocket as well and send commands to the headset. Pressing the "Save" button on the web application records the current gesture. It is sent to the main server application to allow the detailed analysis of gestures in the Visualizer after the study. This was done because it would not have been realistic to record everything the participant does with their hands. The WebSocket then allows the communication with the headset with very little delay, so that important gestures can be saved when the participant is trying something out. The Visualizer also allowed the operator to check if the recording of the gesture data worked during the study. The gesture data was backed up after each participant run. The conversation between the operator and the participants was also recorded on video for analysis. The sessions lasted on average about six minutes.


\section{Results}
$28$ different gesture systems were collected from the video and gesture snapshot information, with five systems appearing twice. Nine gesture systems are not usable for the comparison since they are either not strictly teleportation systems or because they can not be compared to other teleportation methods like a system using a minimap or a type of proxy system. Out of the usable 19 systems, six use a bimanual approach while 13 use one hand. This trend to one-handed gesture systems was also explicitly called for by two participants that expressed some possible downsides of bimanual systems. The participants reported more physical effort and not being able to carry something in one hand while teleporting as a disadvantage.

The usable gestures can be categorized into three large groups of similar gesture systems. 

The largest group of systems are all using at least one hand with the index finger extended to use as a pointing device. In total this method was proposed eight times. One gesture is also using an extended middle finger to make the gesture more distinct (\ref{fig:index2}) but the targeting system otherwise works the same. Six times the system was proposed to have a confirmation step before the teleport is actually performed. According to the participants, this should work by using a "finger gun" type gesture where the thumb is first extended and is then tapped against the base of the index finger to confirm the teleport location as seen in \ref{fig:index}. One user also proposed to gesture an "air tap" with the extended index finger. However, he expressed some concern about the accuracy, since moving the finger to confirm could impact the target selection. Twice the second hand was used as a confirmation step but did not influence the targeting system using the pointing hand.

\begin{figure}[!ht]
    \centering
    \includegraphics[width=0.5\textwidth]{figures/double index.jpg}
    \caption{Pointing gesture with index and middle finger extended.}
    \label{fig:index2}
\end{figure}

\begin{figure}[!ht]
    \centering
    \includegraphics[width=\textwidth]{figures/index.jpg}
    \caption{Index finger pointing gesture with two stages.}
    \label{fig:index}
\end{figure}


The second biggest group of gestures was named six times. All gestures are using a single, open hand with some kind of confirmation step as shown in \ref{fig:palm2}. Four times closing the hand to a fist was proposed as a confirmation step, with the others quickly tapping the index finger and the thumb together to select the target. Another difference between the systems is also the direction the open hand is pointing towards. Four times a ray would start as the normal vector of the palm of the hand, once out of the middle finger. One other system proposed to have the palm upwards, with an arc used as the targeting visualization. The arc would curve in the direction of the middle finger and could be manipulated by changing the height of the hand. Holding the hand up high makes the arc go further and would therefore allow teleports over a larger distance.

\begin{figure}[!ht]
    \centering
    \includegraphics[width=\textwidth]{figures/palm.jpg}
    \caption{Palm gesture with two stages.}
    \label{fig:palm2}
\end{figure}

A third group is named five times and is using only bimanual systems that use a targeting system where the selection vector is produced by some kind of "rangefinder", the user is looking through. This could be a triangle formed by touching both index fingers and both thumbs together. This was proposed four times. Another proposed option is a "diamond" form formed by touching all fingers to their equivalent finger of the other hand. This forms a hole that can be used to look through. This was proposed twice. All but one system also include a confirmation step that is performed by closing other fingers to a half fist or pinching the hole together.

Other honorable mentions are:
\begin{itemize}
    \item Wrist mounted laser pointer
    \item Pointing using a thumb
    \item OK-Gesture with using the middle finger to point, confirmed when opening the sign
    \item Throwing a teleport ball to a target
    \item Drawing a circle in the air that will become the new viewport
\end{itemize}

They were all mentioned only once but could also be interesting to investigate.

\section{Discussion}
The results are very promising for the usability study and for use in all kinds of applications since all the methods implemented to test were also intuitively proposed by the test subjects. Also, the change to add a confirmation step is well supported by the participants. Only 3 out of 19 gesture systems proposed by the participants did not include an explicit explanation for a confirmation step. 

\chapter{Implementation}
The implementation had the following requirements: 

\begin{itemize}
    \item bring different gestures into one environment to be able to compare them
    \item be able to collect statistics for the user study in the future
    \item add a task for the user to perform
    \item the system should have the flexibility be adapted to different gestures or teleport systems in the future
\end{itemize}

To solve the requirements, the following solution was implemented. It is build using a Unity client and a backend server.

\chapter{Backend}
The backend is based on a web server that is connected to a database, as well as a WebSocket manager for instant communication with the Unity client. The database stores the gestures so they can be stored securely, together with logs and collect statistics from the user studies. The code is hosted on a virtual Linux server from the Microsoft Azure cloud service. The service was chosen to get to know it and because there is a student credit program so it can be used for free. The server is accessible through a subdomain of a personal domain I already had. It is secured using an Nginx reverse proxy that manages the traffic coming in from the public internet and passing it on to the internal service. This way it was also possible to encrypt the service using free SSL certificates from the let's encrypt certbot. This would not technically have been necessary but for some reason, Chrome and Firefox defaulted to HTTPS only and would not send HTTP traffic. This might be because the domain without the backend subdomain was already known to the browser and is also encrypted. The encryption took some time to set up however it was made simpler by the use of docker containers and the docker-compose tool that is able to manage multiple containers and the connection between them. The database, the server, the WebSocket manager and the Nginx reverse proxy all have their dedicated containers. This makes the service much easier to spin up, resilient to crashes and adds a layer of security on top. From the web servers logs, I could see many clients trying to connect to publicly known databases and administration endpoints trying to exploit cloud servers, so this seemed like a good idea even though not strictly necessary. 

\section{Services}
There are two backend services running in dedicated containers.

\subsection{SvelteKit Webserver}
The SvelteKit framework was not chosen because it is strictly necessary but rather to get to know it with all its benefits and limitations. It is a framework for websites optimized for load times, SEO and other things not needed in this project so it is quite a bit overkill. On the other hand, it was a good learning opportunity for personal use and made me realize some limitations like that it does not support WebSockets. It is still in the beta phase so this might change in the future. For the database connections, the javascript library Mongoose was used. This library allows easy access to the database since its able to automatically update objects if they are changed. For this project, there are no complicated queries needed, however it was still a nice convenience. 

\subsection{Websocket Service}
The WebSocket service is only running a small application that can relay messages from one client to another. This was only used for the gesture elicitation study to be able to connect a smartphone application to the VR headset, running a C\# websocket client. This service was only used for the gesture elicitation study for now and it is probably not needed for the main study.


\section{Development Process}
New code can automatically be deployed to the server. The server is running a Github Actions Runner Service that listens for new push events on the GitHub repo. If there are changes, the new docker images are built and run using docker-compose. This is based on Github Actions using a self-hosted runner service. This is really easy to configure and increases the development speed by a lot. 

For all of the assets included in the Unity application, Git LFS was used to keep the size of the repository manageable.


\section{Server endpoints}
The server exposes HTTP endpoints that are registered using the SvelteKit framework.

\subsection{Gesture}
The "gesture" endpoint is able to store the gestures and can also provide them to the Unity client in the correct format for serialization. This was tricky to get right since the format always has to be consistent with the C\# code and there needed to be a dedicated Array class just to be able to handle an array of gesture data from the JSON input.

\subsection{Logs}
Debugging the native Android code running on the Oculus Quest is difficult. To help the server is able to collect Logging information from the Unity application. The logs are then displayed on a simple web interface that shows the latest logs. The logs are only stored in memory and not in the database since they are only needed for debugging purposes. 

\subsection{Statistics}
The backend exposes a statistics endpoint. The endpoint excepts JSON data in the form with only loose structural constraints. The Unity application can connect to the endpoint to send any kind of statistics that need to be collected during the study. This information is stored in the database. The tools for the analysis of the data are not part of the project and will be implemented after.  


\section{Visualizations}
With the data collected by the server, it is possible to create visualizations to help understand the data. This was used to find bugs and can later be used to check the users' progress during the study.

\subsection{3D Gesture Visualizer}
The gestures are serialized to a JSON format and saved on the backend. The gestures have a "name" field but are otherwise not human readable. The information about the finger positions is stored in an array of joint positions relative to the base of the hand, together with the normal vector of the palm of the hand. The joints are identified with an id and a field that tells which hand the joint is a part of. All of this can not be debugged or seen at all without the help of a tool. For this reason, a Visualizer was created. It is a browser tool that uses the canvas library p5.js set to a 3D context. It can display the gesture interactively with basic support for model rotation and zooming. The visualization consists of basic shapes like cylinders and cones. They are positioned on the point that the JSON data references. There is also a rotation applied so they connect to the next cylinder and are not just oriented in the same direction. The normal vector of the palm is used to apply the correct rotation to the visualization of the hand. The rotation of the hand is not applied since it is not saved. This is only a small limitation though since it is still easy to see how the gesture was recorded. To tell the fingers apart each part of the hand has a different colour. To change gesture metadata, the Visualizer has some fields that can be updated and are synched automatically to the backend. An example of this can be seen in \ref{fig:vis} The only limitation is that the gesture data can not be changed and has to be re-recorded using the unity application if something changes. This is mostly fine though since recording 3 gestures with one confirmation step each and setting everything up usually takes less than 5 minutes.

\begin{figure}[!ht]
    \centering
    \includegraphics[width=0.8\textwidth]{figures/visualizer.jpg}
    \caption{Example gesture visualization with the available controls.}
    \label{fig:vis}
\end{figure}


\section{Map Visualizer}
To debug the teleportation and to show the users location to the study operator, I created a map visualization that is able to visualize the users' teleport points that are collected by the server. The application is running in the browser and is using the p5.js canvas library but in a 2D context this time. A top-down view of the unity map is used as a background with the teleport points drawn on top. In the future this map could be expanded to analyse the teleport steps that were recorded during the user study. An example of the current implementation can be seen in \ref{fig:map}.

\begin{figure}[!ht]
    \centering
    \includegraphics[width=0.8\textwidth]{figures/map.png}
    \caption{Example map with teleport steps from the start point.}
    \label{fig:map}
\end{figure}


\chapter{Custom Unity Code}

\section{Gesture Recognition System}
To recognize a gesture it has to be compared to a stored gesture. The stored gestures are downloaded from the server when the VR application is started. They can then be compared to the current hand position. This algorithm is adapted from a small script found on Github.  % TODO: cite https://github.com/jorgejgnz/HandTrackingGestureRecorder 
The gestures consist of joint positions and the algorithm compares each of them to the current positions relative to the base of the hand. If a joint is too far of and the distance exceeds a threshold the gesture is skipped. If all joints are within the threshold, all distances are added to a sum. The gesture that has the lowest total distance to the current hand position is then chosen as the recognized gesture. There is also a time-based threshold that checks if the gesture is displayed by the user for at least the time set by the time threshold variable. This prevents accidental activation. The gestures that are part of a set and can currently not be reached are skipped without evaluation so they do not take up any processing time. Another optimization is to skip the evaluation of joints if one hand is not included in the gesture at all. Those optimizations are important because the algorithm is run for every frame and has to look at the 46 joint positions of all hands so it must be as fast as possible.
If a hand is skipped, for the included joints there is a default distance assumed. This is because otherwise a one-handed gesture would always be picked over a two-handed gesture even if the two-handed gesture was invented by the user. Due to the fact that there are always small differences in the tracking or the users' hand that are summed up so, a bimanual gesture would always have a larger distance compared to a skipped hand with zero distance. 
The output of the gesture recognition system is the type of gesture currently displayed by the user, the index of the current gesture if it is part of a set, including information for the teleport service about key positions. The positions are adapted to whatever hand the user prefers.

\section{Teleportation System}
The teleportation system is built in a modular fashion so that it is easy to extend. There is a Teleporter base class that contains most of the methods needed for all the various teleportation methods. If the gesture recognition system detects a new gesture, a new instance of the corresponding teleport class is then created. This is active as long as the gesture does not change or until a teleport is executed. Otherwise, a cleanup method is called and the Teleporter instance is discarded. 

To be able to create a vector from the gesture, the positions of key joints are extracted from the current hand position. The Oculus framework uses an identifier for every joint as well as one for each fingertip and the base of the hand. The Visualizer can show them for debugging (see \ref{fig:fingerIds})

\begin{figure}[!h]
    \centering
    \includegraphics[width=\textwidth/5]{figures/fingerIds.jpg}
    \caption{The numbers of every joint.}
    \label{fig:fingerIds}
\end{figure}

% TODO: describe teleporters and left-handed mode

The three types of target selection methods all function using a ray cast. A ray cast is using a start position to find objects that can be found in the direction of a vector. The vectors the ray follows and the start positions are created differently for each method. 
The palm and the triangle gesture both use a plane created using different joints. The normal vector of the plane is used to get the direction of the ray cast. The start position is a joint in the middle of the hand, in the case of the palm gesture as seen in \ref{fig:palmTracker}. The triangle gesture is using the middle of the hands. It is calculated by linear interpolating from the base of one index finger halfway to the base of the index finger of the other hand. The tip of the triangle is calculated in a similar way using the tips of the index fingers as seen in \ref{fig:triangleTracker}. The selection method using the index gesture is generating a ray cast vector direction from the position of the wrist towards the position of the tip of the index finger as seen in \ref{fig:indexTracker}. All methods are using some amount of smoothing applied to the selection targets to make the output more resilient to tracking errors. This is done by linear interpolating the last target to the next target by a small amount. This way the new target only contributes a limited amount of change. The smoothing values are adapted to fit the accuracy of each gesture. The palm gesture for example requires more smoothing since the confirmation step can otherwise disturb the previously set target.

\begin{figure}[!h]
    \centering
    \includegraphics[width=\textwidth/5]{figures/palm tracker.jpg}
    \caption{The joints used to construct and plane from the palm gesture.}
    \label{fig:palmTracker}
\end{figure}
\begin{figure}[!h]
    \centering
    \includegraphics[width=\textwidth/5]{figures/triangle tracker.jpg}
    \caption{The joints used to construct and plane from the triangle gesture.}
    \label{fig:triangleTracker}
\end{figure}
\begin{figure}[!h]
    \centering
    \includegraphics[width=\textwidth/5]{figures/index tracker.jpg}
    \caption{The joints used to construct a vector from the index gesture.}
    \label{fig:indexTracker}
\end{figure}

If the teleporter has converted a users input gesture to a position in the environment using a ray cast, the location is picked up and a teleport is executed. This requires the deactivation of the Oculus player controller so that a new position for the player and the camera can be set. After the teleport is complete, the player controller is enabled again. Other than the teleport, this process is invisible to the user. 

\section{Study Observers}
In each scene used for the study, there is an observer script that is able to listen to teleports, complete tasks and collect statistics. This information is relayed to the server where it is stored for analysis.

\chapter{Study Environments}
In preparation for the user study, I created a detailed forrest environment and a simple path integration test environment in Unity.



\section{Forrest Environment}

The forrest environment was created for the main part of the study. Here the user can teleport around freely in a large forrest area. It requires at least five teleport jumps across and creates an immersive experience for the user. The application is running natively on the Oculus Quest headset without any connection to a PC that could impact the VR experience in a negative way. The application was optimized to run very well so it is possible to enable the highest hand-tracking polling rate. This makes the experience very natural and improves the immersion. As a preparation, this environment was also used to record the gestures. For this, the Oculus "options" gesture was used to trigger the recording with a time delay. The recording functionality will be disabled in the final version of the application. % map display

During the study the user is instructed to collect four items from all over the map. Each one is hidden in a secret location on the map. The user has to search each one using a different teleportation technique and return the item to the starting location. 


\section{Path Integration Test Environment} % and accuracy?
To test how well the user looses their path integration while teleporting, a simple test environment was created. The environment is kept very simple and only includes a white ground plane and the skybox. The user starts at a point A and is given the task to teleport to a point B that they are free to choose, to turn around and then return to the point A. The environment was created to measure the distance between the actual point A and the point A' the user landed on after two teleportation steps. This distance is recorded together with the distance of the first teleport. The test is run with all gestures in a randomized order. The recorded data will later give insight into the ability to choose an accurate point and how well the path integration works while teleporting.


% - What was realized in the project and which limitations do still exist?
\chapter{Discussion} % (fold)
\label{cha:Discussion}

The quantitative results favour the triangle gesture since it produces the lowest task completion time and the lowest teleport delay. The usability results on the other hand are rarely conclusive, with the dependability being the only UEQ dimension with statistically significant differences between the gestures. All of the gestures are favoured by the participants in some dimension which makes the results hard to interpret. The index gesture for example is getting UEQ results that are better than expected. Users reported being frustrated by the false positive teleport actions but still reported the gesture to be most attractive, efficient, stimulating and novel. A reason for this can be seen in the comments given by the participants. The participants complained about the tracking but still reported the gesture to be fun which is not an attribute given to any of the other gestures. A hypotheses is therefore that the tracking and activation issues are compensated by the gesture being the most fun to use for the user. With improved tracking this could be solved and result in both good usability and good task completion times. For now with the current tracking algorithm, the triangle gesture produces the best results and is reported to be the most dependent. The triangle gesture has the technical benefit of being easy to track because none all of the important joints are visible to the tracking system. It also should have increased accuracy and stability since the algorithm is using four joint positions to generate the targeting vector. This gives the triangle gesture an advantage in some ways but has some drawbacks too. A mechanism had to be added to allow the users to carry an object to be able to complete the first task. This would not be a technical challenge but might be inappropriate for some use cases in real applications. Ideally a gesture for something as important as locomotion would be something that does fade into the background and something does not needs to be worked around for. This makes one-handed gestures more attractive to implement in applications. While the triangle gesture does not produce a significant difference in the overall load experienced by the participants, it is also not reported to be fun. It is the recommended gesture with the current setup but this is likely going to change as the technology improves and is worth further investigation. Applications that require very accurate but infrequent teleportation steps could still benefit from the triangle gesture over the index gesture even in the future but this makes the use-case very limited. Since a standard way to teleport using gestures has not been established it is worth to pick a gesture that is compatible with a variety of environments and tasks. Especially since gestures do not offer the affordance that buttons on a controller do and a user would always need a tutorial to show them how to teleport if there is no standard way established. 


\subsection{Future Work}

As part of the final quantitative questionnaire after all tasks were conducted, the participants were asked to come up with ideas for other gesture systems, similar to the gesture elicitation study. It was to be expected that the answers would be biased by the study up to this point but this task was included because this study included more participants that now also had some experience with how the teleportation using hand-tracking feels like. As expected, the answers were largely based on the previous gestures but some interesting ideas could still be collected. Multiple participants came up with the idea of a one handed version of the triangle gesture system. Also one participant suggested to change the palm gesture systems target direction so the targeting hand would be in a more comfortable position and the direction vector would not change when the hand is closed to confirm the teleport location. The change they proposed would have the hand in an orientation where the thumb would face towards the user with the hand edge pointing in the teleport direction. Both ideas seem like a promising implementation of the participants feedback and could be interesting to investigate in future research.

At the time of writing, a new update to the hand-tracking software of Meta Quest 2, the VR headset that was used to perform both studies, became available. According to the release notes and early reviews from developers, %TODO: cite
this update improves the tracking overall a lot but specifically helps with hand-on-hand interaction. Gestures that rely on this type of interaction could be an interesting target of further research since they were not technically feasible to implement for this work prior to the update. The participants of the gesture elicitation study proposed some gestures that could benefit from this update but not a lot of them. The participants could have been biased  by the current version of the tracking however, since they would experience the tracking cutting out if they tried to make gestures with hand-on-hand interaction as seen in figure \ref{fig:hands20}. Therefore it would be required to perform another gesture elicitation study to confirm if gestures with hand-on-hand interaction are not intuitive for the users. Also the general improvements to the tracking could benefit the usability of the system overall, which would also be an interesting comparison to make in the future. 

\begin{figure}[!ht]
    \centering
    \includegraphics[width=\textwidth]{figures/hands20.png}
    \caption{Update to hand-tracking allowing hand-on-hand interactions, while the previous implementation looses tracking.}
    \label{fig:hands20}
\end{figure}

% - What is - based on the final status of the project - planned for the master thesis? (Outlook to the planned study/data analysis and possibly still necessary revisions of implementation parts)
% - What is the planned schedule in regard to the master thesis?
\chapter{Outlook}

The implemented systems are working well and can be easily adapted for more types of gestures with fast development using the debugging and visualization systems. However there are also some limitations. The gesture detection system is fixed to keyframe like gesture definitions. That means it requires an approximate match of all joints of a hand to be recognized as a new state. That means it would be harder to build systems that calculate a value based on the current hand position. This would require higher identification thresholds or the use of many in-between states, to be able to track for example the distance between a half pinched index finger and thumb. This was not required right now but could be a requirement in the future to be able to give user feedback if the detection threshold for a gesture is almost met. 

A limitation of the development process is that there always needs to be an instance of the backend running and the headset requires an internet connection for the initial download of gestures. However, this could easily be fixed by caching the last gesture set on the local filesystem to be able to have a fallback version. The current setup does require the connection for other reasons anyway though so this was not done.


% =========== Bibliography ===========
\chapter*{References} % Set custom bibliography heading
\renewcommand{\thepage}{\roman{page}} % Use roman page numbers again
\setcounter{page}{\theromanPages} % Set the page counter
\addcontentsline{toc}{chapter}{References} % Add bibliography to table of contents
\defbibheading{bibempty}{} % Remove standard bibliography heading
\printbibliography[heading=bibempty] % Print bibliography and set the heading to the defined empty heading
\end{document}



