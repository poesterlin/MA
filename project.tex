% HCI Group Konstanz Seminar Template
% Version 1.1
%
% ATTENTION: This template was designed to fit the most common requirements for different types of reports (e.g., Seminar to the Project or Project Report). However, if you feel the need to add an extra component or layout feature please talk to your advisor. Please do not change anything on this template unless it is explicitly allowed or agreed with your advisor. However, it is allowed to add packages that do not alter the design/layout of this template.
%

\documentclass[10pt, paper=a4, parskip, oneside]{scrreprt}
\usepackage[T1]{fontenc}
\usepackage[utf8]{inputenc}
\usepackage{hciknseminar}
\usepackage[ngerman, english]{babel} % Spell checking (Using ngerman for german spell checking). The last language is the standard language for the document.
\usepackage{enumitem}
\usepackage{booktabs}
\addbibresource{bibliography.bib}

% =========== Title page ============
\title{Evaluating Gesture Based Teleportation Techniques For Immersive Virtual Environments} % Your title
\type{Master Project Report} % Seminar to the Bachelor/Master Project | Bachelor/Master Project Report
\author{Philip Oesterlin} % Your name
\studentno{01/993546} % Your student number
\group{Human-Computer Interaction} % Do not change this
\department{Computer and Information Science} % Do not change this
\advisor{Jonathan Wieland} % Your advisor
\reviewer{Prof. Dr. Harald Reiterer} % Do not change this
\date{\the\day .~\monthname~\the\year}

% ============= Abstract =============
\abstract{
    Advances in hand-tracking for VR applications allow users to experience virtual environments without controllers. There are already successful implementations of this however many are still stationary experiences. Using hand-tracking requires new locomotion techniques since there are no joysticks or buttons to press to move the users' avatar and there are no established solutions for this problem yet. This report describes the implementation of a flexible teleportation system using hand-tracking. The gesture detection is based on recorded gesture states that can be compared to the current joint positions.
    The project was conducted with the goal of preparing a study environment to compare different gesture-based teleportation systems that were found during a literature review as well as in commercial VR games. To check that the chosen gestures are intuitive, a small gesture elicitation study was performed. 
    The main part of this project is a VR environment, build with Unity and a server connected to the headset for various logging, visualization and debugging tasks.
}

% ============= Document =============
\begin{document}
% Title page
\maketitle
% Abstract
\makeabstract
% Table of contents
\tableofcontents
\addcontentsline{toc}{chapter}{Contents} % Add contents to table of contents
% List of Figures
\listoffigures
\addcontentsline{toc}{chapter}{\listfigurename} % Add list of figures to table of contents
% List of Tables
\listoftables
\addcontentsline{toc}{chapter}{\listtablename} % Add list of tables to table of contents
% Prepare chapters
\clearpage
\setcounter{romanPages}{\value{page}} % Update variable for roman pages
\pagenumbering{arabic} % Turn on page numbering again

% ============= Chapters =============


\chapter{Introduction}
The world of virtual reality (VR) is getting increasingly diverse. All kinds of tools and games are built, pushing the boundary of what is possible to create with current technologies. The computing hardware is also getting more powerful and more consumer-friendly.

\begin{figure}[!h]
    \centering
    \includegraphics[width=0.6\textwidth]{figures/handtracking.jpg}
    \caption{Meta Quest hand-tracking announcement in 2019 \cite{annoucment}}
    \label{fig:quest}
\end{figure}

Standalone headsets that do not require a powerful computer linked over a cable are cheaper and can even be more immersive since the user does not has to keep track of the cable. The next step on the path of simplifying the experience could be to make it possible to use VR without controllers. Controllers will still have their place for applications that require very acuate tracking, lots of different types of inputs or where it makes sense to hold something like a gun or paintbrush. Many experiences do not require this, however. 
Especially for applications and tools used to collaborate, enabling users to enhance their communication with gestures might be helpful. Use-cases like this are where hand-tracking will allow the implementation of immersive, natural interfaces. This can allow anybody to simply put a headset on and start interacting with the virtual world without first being told how to grab something or how to activate a button, as shown in figure \ref{fig:example}.

\begin{figure}[!h]
    \centering
    \includegraphics[width=0.6\textwidth]{figures/wave.JPG}
    \caption{Waving to talk to bots in Vacation Simulator}
    \label{fig:waving}
\end{figure}

For now, with the current technology, there are still lots of changes to solve. Owlchemy Labs, the developers for the popular game ``Vacation Simulator'' share their experience adding hand-tracking support on their blog and write: ``It’s very easy to occlude fingers, for hands to leave the tracking area, and for velocity to be high enough that no accurate data is available.'' \cite{VacSimBlog}. This is an examples for the limitations hand-tracking has, in terms of accuracy and the number of possible distinct inputs. When converting an existing application to support hand-tracking or creating a new one from scratch, the inputs are a big challenge to get right. Users will first apply the mental model built during decades of experience using their hands in the real world. If this experience translates to the virtual world, it can be great experience. ``In Vacation Simulator you can wave at bots to start conversations, and it’s fun and intuitive. This was one place where hand tracking made the gameplay even better. With waving there is no abstraction at all in what you are doing. The game feel is unmatched.'' write the Owlchemy Labs developers further. This interaction can be seen in figure \ref{fig:waving}. However, there are going to be some differences between the real world interaction and the virtual word, that make the experience less immersive. This raises the question of what to do if the virtual world allows users to do more with their hands than in the real world. 

\begin{figure}[!h]
    \centering
    \includegraphics[width=\textwidth]{figures/examplegestures.png}
    \caption{Example of users interacting with objects in VR using hand-tracking. \cite{Han}}
    \label{fig:example}
\end{figure}

One simple way where VR technology can expand on what is possible in the real world is scale. The area a user is physically able to explore and walk around comfortably is limited. For most casual users of VR, that might be the size of their living room. In VR, this is not a limitation. There, a user can be given the ability to fly, teleport or use any number of so-called locomotion techniques to increase their level of comfort and allow them to go wherever they want. While not completely understood, the disconnect between the real world's movement and VR is likely one source of motion sickness. This makes locomotion a necessary mechanic to implement in large scale VR experiences and a very divisive one since it can impact the users' immersion and level of comfort a lot. VR experiences only offering continuous locomotion can, for some users that are more prone to get motion sick, only be usable for a few minutes, only with a limited field of view or not useable at all. On the other hand, users that are used to continuous navigation do not get motion sick as quickly or even at all report being frustrated by having to use another, less immersive locomotion mechanic that might get in the way a lot more. So while teleportation based locomotion is not needed for everyone, it allows most people to at least use large scale VR experiences comfortably without getting sick. This makes it the first and sometimes only mechanic that is implemented, and so it should be as well understood and implemented as possible. Especially when in combination with hand-tracking and gesture-based teleportation, there is only minimal research done so far.

This work expands the previously done research on how locomotion or, more specifically, teleportation should work using hand-tracking. Teleportation is the most popular form of locomotion in VR. When using controllers, locomotion is usually controlled using a joystick or button, and even though there are slight differences between each game, you can usually figure out quickly how to control it. Without controllers, there is no conventional way to control teleportation. The user can also not easily hit all the buttons until some feedback can lead them in the right direction. VR applications based on hand-tracking will, therefore, still have a learning curve for new users. If there is a standard way to use teleportation established, it should be based on research. 

The goal of this evaluation is to recommend how the ideal gesture would look like that allows for comfortable, intuitive and accurate gesture-based teleportation. To be able to do that, three gestures that were collected in a gesture elicitation study \cite{elicitation} were implemented in a study prototype. The gestures were then compared in a controlled lab study. The results of this comparison are reported, evaluated, and discussed in this thesis. Finally, a recommendation is given to future developers and designers that want to integrate hand-tracking into their VR environment that requires a teleportation gesture.

The thesis follows this structure. The related work is summarized first with a broad look at the subjects involved. The next chapter describes the gesture elicitation study. Chapter 4 then gives an overview of the implementation details, followed by the comparative study and its results. The discussion then includes an evaluation of the results and a look into future work. The conclusion, the last chapter, provides a summary of the thesis.



% - How do the results of the seminar work inform the project?
% - Which requirements need to be addressed from a technical point of view to realize the concept that was derived in the seminar work (interaction concept/study setting)?
\chapter{Results from the Literature Research}

There is not much research focused on alternative locomotion techniques in VR that do not require controllers. It is a new and experimental field of study.

\section{Tracking Technology}
Locomotion systems using hand-tracking traditionally required external tracking devices like a leap motion sensor mounted to the VR headset. %TODO: cite
This works for a prototype but can not be expected to be something an end-user is able to set up. In the studies conducted using the leap motion sensor also reported users complaining about the sensors field of view. %TODO: cite
The sensor has a 140° field of view and therefore can see the environment in front of the headset well, however it is not able to track hands if they are held below the headset or next to the users body, where it would be more natural and ergonomic. A much more usable implementation of hand-tracking is implemented by the Oculus Quest headsets. %TODO: cite
The headset is using internal tracking using up to 4 cameras that are able to detect the users hands using a trained neural network. %TODO: cite
Even though the cameras only produce black and white images and sometimes only one camera is able to see the hand, the tracking works remarkably well. There are however major limitations since the detection network is not trained to detect hand-to-hand interactions and therefore cuts out entirely until the user separates their hands again. The tracking is also sometimes inaccurate if some fingers are obstructed by others. In most cases this is not very visible to the user since the fingers are also obstructed from their perspective, it is however visible in the tracking information recorded by the headset and has to be taken into account. 

\section{Requirements for Teleportation Systems}
%TODO

\section{Gestures}
The literature has come up with a handful of gestures and ways to study them in a laboratory environment. The games industry however has implemented some fully usable systems for actual users. Both types of sources were taken into account to produce this list of comparable gestures:

\begin{itemize}
    \item One-Handed Palm Gesture: %TODO: cite
    Researchers propose a teleportation gesture that is using a single, wide open hand. The target location is selected using a ray starting as the normal vector of the palm of the hand. The teleportation is then executed after 1.5 seconds.

    \item One-Handed Index Gesture: %TODO: cite
    Researchers propose a teleportation gesture that is using one hand with the index finger extended to use as a pointing device. To select a point, a user can just point at it. The teleportation is then executed after 1.5 seconds.
    
    \item Two-Handed Palm Gesture: %TODO: cite
    Same targeting gesture as the One-Handed Index Gesture. The second hand is used to confirm the teleport by opening the hand.

    \item One-Handed Index Gesture: %TODO: cite
    Same targeting gesture as the One-Handed Index Gesture. The second hand is used to confirm the teleport by extending the index finger.

    \item Triangle Gesture: %TODO: cite
    A game called Elixir which is a short hand-tracking demo for the Oculus Quest uses a bimanual gesture. Both hands form a triangle connecting both thumbs and index fingers together. This can be used to target a point on the floor. To confirm the target location, the user pinches thumbs and index fingers together.
    
    \item Pull Gesture: %TODO: cite
    A game called Vacation Simulator has one of the best implementations of hand-tracking found in games so far. For the locomotion it is using a one-handed pulling gesture. In the game this is only used for low resolution target selection. 

\end{itemize}

After my own subjective testing using a prototype of the final application, the systems using a time delay were found to be disorienting. Because the teleportation can not be confirmed explicitly it is not transparent to the user when it is going to be executed. The locomotion systems of the first four gestures are also very similar and lead to very similar results in previous testing. Because of this, I decided to combine the one- and two-handed systems. The result is two one-handed systems that have a confirmation step to execute the teleportation. To get feedback on how intuitive this change was, a gesture elicitation study was performed. 

\section{Testing Methodology}
The internal test game studios are using are not known, so the sources for testing methodology are limited. 

% - How was the project realized? (methodology)
% - Which previous work (e.g., related to interaction concepts or algorithms), and other available products (software, hardware, and other materials) can help to address the requirements to which extent? (comparative analysis and clear reasons for choices that were made to realize the project)
% Introduction

% - What is the motivation for the project?
% - How is the project report structured? (Overview for the different sections)


\section{Gesture Elicitation Study}
To be able to evaluate the intuitiveness of the implemented gestures, a Gesture Elicitation Study was performed. The study was conducted with 5 participants, aged ... The participants where instructed to come up with teleportation techniques and to explain them in detail. To understand the limitations of the hand-tracking they where wearing the Oculus Quest device with hand-tracking enabled. This way it is easy to tell which gesture is tracking well and what is not detected by the tracking cameras. A simple environment with only a plain on the bottom, the skybox and the users tracked virtual hands was used. The environment was created so that the focus lies on the hands and the gestures. During the study, the headset was connected to the websocket relay application. Using a simple web interface, the study operator could connect to the websocket as well and send commands to the headset. Pressing "save" button on the web application records the current gesture. It is send to the main server application to allow the detailed analysis of gesture in the Visualizer after the study. This was done because it would not have been realistic to record everything the participant does with their hands. The websocket then allows the communication with the headset with very little delay, so that important gestures can be saved when the participant is trying something out. The Visualizer also allowed the operator to check if the recording of the gesture data worked during the study. The gesture data was backed up after each participants run. The conversation between the operator and the participants was also recorded on video for analysis.

\chapter{Implementation Requirements}
The implementation had four major requirements. First, the different gestures selected should be combined into one immersive environment. They originate from different sources and have not been directly comparable before. The level of detail should be fairly high. Second, a task for the user should be implemented that is similar to what a teleportation task in a game is like. A fun task in an immersive world will keep study participants engaged. Third it should be possible to collect statistics about what the user does. This is important for the user study that will be performed later. The last requirement is to keep the system flexible so it could be adapted to different gesture recognition tasks or to other teleport systems in the future.

To solve the requirements, the following solution was implemented. It is build using a Unity client and a backend server.

\chapter{Backend}
The backend is based on a webserver that is connected to a database, as well as a websocket manager for instant communication with the Unity client. The database stores the gestures so they can be stored securely, together with logs and collected statistics from the user studies. The code is hosted on a virtual linux server from the Microsoft Azure cloud service. The service was chosen to get to know it and because there is a student credit program so it can be used for free. The server is accessible through a subdomain of a personal domain I already had. It is secured using an nginx reverse proxy that manages the traffic coming in from the public internet and passing it on to the internal service. This way it was also possible to encrypt the service using free ssl certificates from the lets encrypt certbot. This would not technically have been necessary but for some reason Chrome and Firefox defaulted to https only and would not send http traffic. This might be because the domain without the backend subdomain was already known to the browser and is also encrypted. The encryption took some time to setup however it was made simpler by the use of docker containers and the docker-compose tool that is able to manage multiple containers and the connection between them. The database, the server, the websocket manager and the nginx reverse proxy all have their own dedicated containers. This makes the service much easier to spin up, resilient to crashes and adds a layer of security on top. From the web servers logs I could see many clients trying to connect to publicly known database and administration endpoints trying to exploit cloud servers, so this seamed like a good idea even though not strictly necessary. 

\section{Services}
There are two backend services running in dedicated containers.

\subsection{SvelteKit Webserver}
The SvelteKit framework was not chosen because it is strictly necessary but rather to get to know it with all its benefits and limitations. It is a framework for websites optimized for load times, SEO and other things not needed in this project so its quite a bit overkill. On the other hand it was a good learning opportunity for personal use and made me realize some limitations like that it does not support websockets. Its still in the beta phase so this might change in the future. For the database connections, the javascript library Mongoose was used. This library allows easy access to the database since its able to automatically update objects if they are changed. For this project there are no complicated queries needed, however it was still a nice convenience. 

\subsection{Websocket Service}
The websocket service is only running a small application that can relay messages from one client to another. This was only used for the gesture elicitation study to be able to connect a smartphone application to the VR headset, running a C\# websocket client. This service was only used for the gesture elicitation study for now and its probably not needed for the main study.


\section{Development Process}
New code can automatically be deployed to the server. The server is running a Github Actions Runner Service that listens for new push events on the github repo. If there are changes, the new docker images are build and run using docker-compose. This is based on Github Actions using a self-hosted runner service. This is really easy to configure and increases the development speed by a lot. 

For all of the assets included in the Unity application Git LFS was used to keep the size of the repository manageable.


\section{Server endpoints}
The server exposes http endpoints that are registered using the SvelteKit framework.

\subsection{Gesture}
The "gesture" endpoint is able store the gestures and can also provide them to the Unity client in the correct format for serialization. This was tricky to get right since the format always has to be consist to the C\# code and there needed to be a dedicated Array class just to be able to handle an array of gesture data from the JSON input.

\subsection{Logs}
Debugging the native Android code running on the Oculus Quest is difficult. To help the server is able to collect Logging information from the Unity application. The logs are then displayed on a simple web interface that shows the latest logs. The logs are only stored in memory and not in the database since they are only needed for debugging purposes. 

\subsection{Statistics}
The backend exposes a statistics endpoint. The endpoint excepts JSON data in the form with only loose structural constraints. The Unity application can connect to the endpoint to send any kind of statistics that need to be collected during the study. This information is stored in the database. The tools for the analysis of the data are not part of the project and will be implemented after.  


\section{Visualizations}
With the data collected by the server it is possible to create visualizations to help understand the data. This was used to find bugs and can later be used to check the users progress during the study.

\subsection{3D Gesture Visualizer}
The gestures are serialized to a JSON format and saved on the backend. The gestures have a "name" field but are otherwise not human readable. The information about the finger positions is stored in an array of joint positions relative to the base of the hand, together with the normal vector of the palm of the hand. The joints are identified with an id and a field that tells which hand the joint is a part of. All of this can not be debugged or seen at all without the help of a tool. For this reason a Visualizer was created. Its a browser tool that uses the canvas library p5.js set to a 3D context. It can display the gesture interactively with basic support for model rotation and zooming. The visualization consists of basic shapes like cylinders and cones. They are positioned on the point that the JSON data references. There is also a rotation applied so they connect to the next cylinder and are not just oriented in the same direction. The normal vector of the palm is used to apply the correct rotation to the visualization of the hand. The rotation of the hand is not applied since it is not saved. This is only a small limitation though since it is still easy to see how the gesture was recorded. To tell the fingers apart each part of the hand has a different color. To change gesture meta data, the Visualizer has some fields that can be updated and are synched automatically to the backend. The only limitation is that the gesture data can not be changed and has to be re-recorded using the unity application if something changes. This is mostly fine though since recording 3 gestures with one confirmation step each and setting everything up usually takes less than 5 minutes.


\section{Map Visualizer}
To debug the teleportation and to show the users location to the study operator, I created a map visualization that is able to visualize the users teleport points that are collected by the server. The application is running in the browser and is using the p5.js canvas library but in a 2D context this time. A top down view of the unity map is used as a background with the teleport points drawn on top.


\chapter{Custom Unity Code}

\section{Gesture Recognition System}
To recognize a gesture it has to be compared to a stored gesture. The stored gestures are downloaded from the server when the VR application is started. They can then be compared to the current hand position. This algorithm is adapted from a small script found on Github.  % TODO: cite https://github.com/jorgejgnz/HandTrackingGestureRecorder 
The gestures consists of joint positions and the algorithm compares each of them to the current positions relative to the base of the hand. If a joint is too far of and the distance exceeds a threshold the gesture is skipped. If all joints are within the threshold, all distances are added to a sum. The gesture that has the lowest total distance to the current hand position is then chosen as the recognized gesture. There is also a time based threshold that checks if the gesture is displayed by the user for at least the time set by the time threshold variable. This prevents accidental activation. The gestures that are part of a set and can currently not be reached are skipped without evaluation so they do not take up any processing time. Another optimization is to skip the evaluation of joints if one hand is not included in the gesture at all. Those optimizations are important because the algorithm is run for every frame and has to look at the 46 joint positions of all hands so it is important that it is as fast as possible.
If a hand is skipped, for the included joints there is a default distance assumed. This is because otherwise a one handed gesture would always be picked over a two handed gesture even if the two handed gesture was indented by the user. This is because there are always small differences in the tracking or the users hand and they are summed up so that it would always loose to a skipped hand with zero distance. 
The output of the gesture recognition system is the type of gesture currently displayed by the user, the index of the current gesture if its part of a set, including information for the teleport service about key positions.

\section{Teleportation System}
The teleportation system is build in a modular fashion so that it is easy to extend. There is a Teleporter base class that contains most of the methods needed for all the various teleportation methods. If the gesture recognition system detects a new gesture, a new instance of the corresponding teleport class is then created. This is active as long as the gesture does not change or until a teleport is executed. Otherwise a cleanup method is called and the Teleporter instance is discarded. If the teleporter has converted a users input gesture to a position in the environment using a ray cast, the location is picked up and a teleport is executed. This requires the deactivation of the Oculus player controller, so that a new position for the player and the camera can be set. After the teleport is complete, the player controller is enabled again. Other than the teleport, this process is invisible to the user. 

\section{Study Environments}
In preparation for the user study, I created a detailed forrest environment and a simple path integration test environment in Unity.



\subsection{Forrest Environment}

The forrest environment was created for the main part of the study. Here the user can teleport around freely in a large forrest area. It requires at least five teleport jumps across and creates an immersive experience for the user. The application is running natively on the Oculus Quest headset without any connection to a PC that could impact the VR experience in a negative way. The application was optimized to run very well so it is possible to enable the highest hand-tracking polling rate. This makes the experience very natural and improves the immersion. As a preparation, this environment was also used to record the gestures. For this, the Oculus "options" gesture was used to trigger the recording with a time delay. The recording functionality will be disabled in the final version of the application. % map display

During the study the user is instructed to collect four items from all over the map. Each one is hidden in a secret location on the map. The user has to search each one using a different teleportation technique and return the item to the starting location. 


\subsection{Path Integration Test Environment} % and accuracy?
To test how well the user looses their path integration while teleporting, a simple test environment was created. The environment is kept very simple and only includes a white ground plane and the skybox. The user starts at a point A and is given the task to teleport to a point B that they are free to choose, to turn around and then return to the point A. The environment was created to measure the distance between the actual point A and the point A' the user landed on after two teleportation steps. This distance is recorded together with the distance of the first teleport. The test is run with all gestures in a randomized order. The recorded data will later give insight into the ability to choose an accurate point and how well the path integration works while teleporting.


% - What was realized in the project and which limitations do still exist?
\chapter{Discussion} % (fold)
\label{cha:Discussion}

The quantitative results favour the triangle gesture since it produces the lowest task completion time and the lowest teleport delay. The usability results on the other hand are rarely conclusive, with the dependability being the only UEQ dimension with statistically significant differences between the gestures. All of the gestures are favoured by the participants in some dimension which makes the results hard to interpret. The index gesture for example is getting UEQ results that are better than expected. Users reported being frustrated by the false positive teleport actions but still reported the gesture to be most attractive, efficient, stimulating and novel. A reason for this can be seen in the comments given by the participants. The participants complained about the tracking but still reported the gesture to be fun which is not an attribute given to any of the other gestures. A hypotheses is therefore that the tracking and activation issues are compensated by the gesture being the most fun to use for the user. With improved tracking this could be solved and result in both good usability and good task completion times. For now with the current tracking, the triangle gesture produces the best results and is reported to be the most dependent. 

The teleport trajectories can not be quantified  teleport trajectories. 

\subsection{Future Work}

As part of the final quantitative questionnaire after all tasks were conducted, the participants were asked to come up with ideas for other gesture systems, similar to the gesture elicitation study. It was to be expected that the answers would be biased by the study up to this point but this task was included because this study included more participants that now also had some experience with how the teleportation using hand-tracking feels like. As expected, the answers were largely based on the previous gestures but some interesting ideas could still be collected. Multiple participants came up with the idea of a one handed version of the triangle gesture system. Also one participant suggested to change the palm gesture systems target direction so the targeting hand would be in a more comfortable position and the direction vector would not change when the hand is closed to confirm the teleport location. The change they proposed would have the hand in an orientation where the thumb would face towards the user with the hand edge pointing in the teleport direction. Both ideas seem like a promising implementation of the participants feedback and could be interesting to investigate in future research.

At the time of writing, a new update to the hand-tracking software of Meta Quest 2, the VR headset that was used to perform both studies, became available. According to the release notes and early reviews from developers, %TODO: cite
this update improves the tracking overall a lot but specifically helps with hand-on-hand interaction. Gestures that rely on this type of interaction could be an interesting target of further research since they were not technically feasible to implement for this work prior to the update. The participants of the gesture elicitation study proposed some gestures that could benefit from this update but not a lot of them. The participants could have been biased  by the current version of the tracking however, since they would experience the tracking cutting out if they tried to make gestures with hand-on-hand interaction as seen in figure \ref{fig:hands20}. Therefore it would be required to perform another gesture elicitation study to confirm if gestures with hand-on-hand interaction are not intuitive for the users. Also the general improvements to the tracking could benefit the usability of the system overall, which would also be an interesting comparison to make in the future. 

\begin{figure}[!ht]
    \centering
    \includegraphics[width=\textwidth]{figures/hands20.png}
    \caption{Update to hand-tracking allowing hand-on-hand interactions, while the previous implementation looses tracking.}
    \label{fig:hands20}
\end{figure}

% - What is - based on the final status of the project - planned for the master thesis? (Outlook to the planned study/data analysis and possibly still necessary revisions of implementation parts)
% - What is the planned schedule in regard to the master thesis?
\chapter{Outlook}

The implemented systems are working well and can be easily adapted for more types of gestures with fast development using the debugging and visualization systems. To be able to conduct the study, a way to switch between the environments has to be implemented and tested well. This is important to get right since this would have to include some way of informing the user how the teleportation gestures work. This needs to be reliable otherwise it could frustrate the user or require the conductor of the study to step in and help.  

After the future study, the collected information has to be analysed so there needs to be a way to generate various diagrams from the data.


% =========== Bibliography ===========
\chapter*{References} % Set custom bibliography heading
\renewcommand{\thepage}{\roman{page}} % Use roman page numbers again
\setcounter{page}{\theromanPages} % Set the page counter
\addcontentsline{toc}{chapter}{References} % Add bibliography to table of contents
\defbibheading{bibempty}{} % Remove standard bibliography heading
\printbibliography[heading=bibempty] % Print bibliography and set the heading to the defined empty heading
\end{document}



